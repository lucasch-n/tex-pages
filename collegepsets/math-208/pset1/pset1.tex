\documentclass{amsart}
\usepackage{../../../lucas}
\usepackage{amsmath, amssymb}
\usepackage{graphicx}

\title{Problem Set 1}
\author{Lucas\ Chen}
\date{\today}
\begin{document}

\maketitle

Problems: Rudin Chapter 6, 1-9, 15-17

\section*{Solutions}

\subsection*{Problem 1}
Since $f$ is constant outside of $x_0$ it is only discontinuous at $x_0$, and since $\alpha$ is continuous at $x_0$ it is continuous at each of $f$'s (finitely many) discontinuities and $f$ is Stieltjes integrable.

\medskip \noindent To prove the integral is $0$, for $\epsilon>0$ we take $\delta$ for which $d(x, x_0)<\delta$ implies $d(\alpha(x), \alpha(x_0))<\epsilon/2$. Then take the partition $x_0-\delta, x_0+\delta$ and we have $\alpha(x_0+\delta)-\alpha(x_0-\delta)<\epsilon$, implying that the upper sum of the partition is less than $\epsilon$. Since the lower sum of all partitions is $0$, this implies that $\int_a^b f\, d\alpha=0$. 

\subsection*{Problem 2}
Take $f(x_0)\neq 0$ for some $x_0\in [a,b]$, and WLOG assume $f(x_0)>0$. Since $f$ is continuous $\exists$ $\delta$ where $d(x, x_0)<\delta$ implies $d(f(x_0), f(x))<f(x_0)/2$. Then $f([x_0-\delta, x_0+\delta])$ is bounded below by $f(x_0)/2$ and the lower sum is bounded below by $f(x_0)\delta>0$, leading to a contradiction. 

\subsection*{Problem 3}

\medskip \noindent (a) Assume first that $f(0+)=f(0)$. Take the set of partitions $(\{0,a\}), a>0$. Then for $\epsilon>0$ take the $\delta$ such that $d(f(0), f(x))<\epsilon/3$ for $x\in (0, \delta)$, and the partition $P_{\epsilon}=\{0, \delta\}$. Then $L(f, P_{\epsilon}, \beta_1)=\inf(f)_{[0,\delta]}\geq f(0)-\epsilon/3$ and $U(f,P_{\epsilon},\beta_1)=\sup(f)_{[0,\delta]}\leq f(0)+\epsilon/3$. This yields $U(f,P_{\epsilon},\beta_1)-L(f,P_{\epsilon},\beta_1)\leq 2\epsilon/3<\epsilon$, implying $f$ is Stieltjes integrable and $\int f\, d\beta_1= f(0)$.

\medskip \noindent Now assume $f(0+)\neq f(0)$. If this is the case, then we know that $f(0)$,
$\limsup_{x\to 0+} f$, and $\liminf_{x\to0+}f$ are not all equal. WLOG assume $f(0)\neq\limsup_{x\to0+}f$. We note that for any $\delta$ and the interval $(0, \delta)$, for any $\epsilon$ there exists an $x\in(0, \delta)$ where $d(\limsup_{x\to0+}f, f(x))<\epsilon$. Take $\epsilon< \frac{|f(0)-\limsup_{x\to0+}f|}{2}$. Then for any interval $(0,\delta)$ there exists $x$ where $d(f(0), f(x))>\frac{|f(0)-\limsup_{x\to0+}f|}{2}$. For any partition $P$ we know there exists an interval that contains $0$ and a number greater than $0$: if the partition divides at 0 then we use the interval to the right of $0$. Then on this interval the infinum and supremum are separated by $f(0)$ and $f(x)$ with $x$ chosen above, and the upper and lower sums cannot converge to one another.

\bigskip \noindent (b) We replace the condition above with $f(0-)=f(0)$. The proof is exactly the same except we consider the left-hand limits rather than the right-hand ones and take a closed interval containing $0$ and some negative number.

\bigskip \noindent (c) Assume $f$ is discontinuous at $0$. Then we know either $f(0+)$ or $f(0-)\neq f(0)$. We take the same argument as (a) for intervals that contain $0$ and values on both sides, where the sums are separated by $\frac{|f(0)-\limsup_{x\to0}f|}{2}$. For partitions that contain $0$ we can divide the Riemann sums to the left and right of $0$. The left sums are separated by $\frac{f(0)-\limsup_{x\to0-}f|}{4}$ and the right sums are separated by $\frac{|f(0)-\limsup_{x\to0+}f|}{4}$ (replace the $\limsup$ with $\liminf$ if $\liminf$ is different from $f(0)$). Since these sums are not the same the function is not Riemann integrable.

\medskip \noindent Now assume $f$ is continuous at $0$. Then for $\epsilon>0$ take $\delta$ where $d(f(0), f(x))<\epsilon/3$ once again, and the partition $P_{\epsilon}=\{-\delta, \delta\}$. Then once again $U(f,P_{\epsilon},\beta_3)-L(f,P_{\epsilon},\beta_3)\leq 2\epsilon/3$ and $L(f,P_\epsilon,\beta_3)\geq f(0)-\epsilon/3$, $U(f,P_\epsilon,\beta_3)\leq f(0)+\epsilon/3$ implies $\int f\,d\beta_3=f(0)$.

\bigskip \noindent (d) Continuity satisfies the conditions of (a), (b), (c). Proof is trivial.

\subsection*{Problem 4}
Since the rationals are dense the upper sum will always be equal to $b-a$ and the lower sum will always be equal to $0$ for any partition and the function is not Stieltjes integrable.

\subsection*{Problem 5}
Does not follow. Consider the function $f$ where $f(x)=1$ for all rationals and $f(x)=-1$ for all irrationals. However, since we know $f$ is a bounded real function on $[a.b]$ we have $f=\sqrt[3]{f^3}$ and since $\sqrt[3]{x}$ is a continuous real function we have that $f$ is Stieltjes integrable.

\subsection*{Problem 6}
Take $\epsilon>0$. Take the total bounds of the function $m<f(x)<M$. Now consider the smallest $n$ where $(M-m)(\frac{2}{3})^n<\epsilon/3$. Consider the $n$th iteration of the intermediate Cantor set, which has combined horizontal length $(\frac{2}{3})^n$. 

\medskip \noindent Take an open cover enclosing this partial Cantor set, and extending beyond it with less than $\epsilon/3(M-m)$ total measure (we cannot place the partition points at the exact bounds of the Cantor set because $f$ is not guaranteed to be continuous there.) Now if we remove this cover of the partial Cantor set, we are left with a compact set which $f$ is continuous over, meaning $f$ is Stieltjes integrable over this compact set. Then there is a further partition of this compact set whose lower and upper sums differ by less than $\epsilon/3$.

\medskip \noindent We combine this partition with the points that bound our open cover, whose lower and upper sums must differ by a maximum of $2\epsilon/3$ since their total horizontal length is less than $2\epsilon/3(M-m)$. Thus the difference between the upper and lower sums of the whole combined partition must be less than $\epsilon$, and we conclude that $f$ is Stieltjes integrable over $[0,1]$. 

\subsection*{Problem 7}
(a) Take $\epsilon>0$. Since $f$ is Stieltjes integrable we know it is bounded over $(0, 1]$ and we can extend the bounds to include $f(0)$: call the bounds $M>f(x)>m$. Then take $c<\min(\epsilon/4(M-m),|\epsilon/4M|,|\epsilon/4m|)$: we have that since $f$ is Stieltjes integrable over $[c,1]$ $\exists$ a partition $P$ where $U(f,P)-L(f,P)<\epsilon/4$. Then extending the partition $P$ to $[0,1]$ we add just the break point at $0$, yielding the partition $P'$: we have \[U(f,P')-L(f,P')<(U(f,P)+Mc)-(L(f,P)+mc)<\epsilon/2\]
Then since $L(f,P')\leq\int_0^1f\,dx\leq U(f,P')$ and $L(f,P)\leq\int_c^1f\,dx\leq U(f,P)$ we have $d(\int_c^1f\,dx, L(f,P))<\epsilon/4$, $d(L(f,P), L(f,P'))=|mc|<\epsilon/4$, and $d(\int_0^1f\, dx, L(f,P'))<\epsilon/2$. This implies $d(\int_c^1f\,dx,\int_0^1f\,dx)<\epsilon$ and we are done.

\medskip \noindent (b) Take \[f(x)= \begin{cases}
	0 & x=0 \\
	\frac{(-2)^n}{n} & x\in\left(\frac{1}{2^n},\frac{1}{2^{n-1}}\right]
\end{cases}\] Then the improper integral is the alternating harmonic series and the integral of the absolute value of the function is the normal harmonic series which is divergent.

\subsection*{Problem 8}
We have (if it exists) $\int_1^{\infty}f\,dx = \lim_{a\to\infty}\int_1^af\,dx = \lim_{a\to\infty}\sum_{n=1}^a \int_n^{n+1}f\,dx$. We note that for each $n$ the partition $\{n, n+1\}$ yields upper and lower sums $f(n+1)$ and $f(n)$ and thus $f(n)<\int_n^{n+1}f\,dx<f(n+1)$. Assume the improper integral converges (and f is monotonic and therefore positive). Then by series comparison $\sum_{n=0}^{\infty} f(n)$ converges. Now assume $\sum_{n=0}^{\infty}f(n)$ and therefore $\sum_{n=1}^{\infty}f(n)$ converges. Since $f$ is monotonically decreasing each of $f(n)$ must be nonnegative or it would not converge, and thus $f$ in general must be nonnegative. We know $\sum_{n=0}^{\infty}\int_n^{n+1}f\,dx$ converges via series comparison, and since $f$ is everywhere nonnegative $\int_1^nf\,dx\leq\int_1^af\,dx\leq\int_1^{n+1}f\,dx$ for $n\leq a\leq n+1$, meaning $\int_1^{\infty}f\,dx$ converges.

\subsection*{Problem 9}
(7) We take $F$, $G$ differentiable on $[c, 1]$ for each $c>0$, and $F'=f$, $G'=g$ Stieltjes integrable over these intervals. Then if $G$ is differentiable and $g$ Stieltjes integrable on $[0,1]$ we have \[\int_0^1Fg\,dx=\lim_{c\to0}\int_c^1Fg\,dx=F(1)G(1)-\lim_{c\to0}F(c)G(c)-\lim_{c\to0}\int_c^1fG\,dx\]

\medskip \noindent \textbf{Proof}: This equation is achieved by taking the limit of both sides of a standard integration by parts equation, and then applying 7(a). Note that all terms are Stieltjes integrable since they are the products of Stieltjes integrable functions.

\medskip \noindent We note that the expression $F(1)G(1)-F(0)G(0)-\int_0^1 fG\,dx$ is only equal here if $F$ is differentiable and $f$ Stieltjes integrable as in the conditions for both integration by parts and 7(a). 

\medskip \noindent For the equation we have \[\int_0^{\infty}\frac{\cos x}{1+x}\,dx=\lim_{c\to\infty}\frac{\sin c}{1+c}-\frac{\sin 0}{1}-\int_0^{\infty}-\frac{\sin x}{(1+x)^2}\,dx=\int_0^{\infty}\frac{\sin x}{(1+x)^2}\,dx\]
To prove $\int_0^\infty \frac{\sin x}{(1+x)^2}\,dx$ is absolutely convergent we note that $\left|\frac{\sin x}{(1+x)^2}\right|\leq \frac{1}{(1+x)^2}$ whose integral we know converges to $1$. 

\medskip \noindent To show $\int_0^\infty\left|\frac{\cos x}{1+x}\right|\,dx$ is not absolutely convergent we consider the values of the integrand at $\frac{3\pi}{4}+\pi n$ for $n\in \mathbb{N}$. This yields us $\frac{\sqrt{2}}{2(3\pi/4+1+\pi n)}$i. The sum of this sequence over all $n$ diverges via a harmonic series argument (we can factor out a constant so the denominator increases by 1 for each $n$). We note that the integrand is always greater than this value over $[\frac{\pi}{4}+\pi n, \frac{3\pi}{4}+\pi n)$ and so multiplying the series by $\pi/2$ bounds the integral from below, meaning it diverges.


\subsection*{Problem 15}
We use integration by parts for the first part. \[1=\int_a^bf^2(x)\,dx=bf^2(b)-af^2(a)-\int_a^b2xf(x)f'(x)\,dx.\]
\[\implies \int_a^bxf(x)f'(x)\,dx = -1/2.\]
We take the inner product (as described in class) of $\int f_1f_2\,dx$. Then taking $f_1=xf(x)$ and $f_2=f'(x)$ we have by the Cauchy-Schwartz inequality: \[1/4=\left[\int_a^bxf(x)f'(x)\,dx\right]^2\leq \int_a^b\left(f'(x)\right)^2\,dx\int_a^bx^2f^2(x)\,dx\]

To disprove equality we must prove that the solutions of $f'(x) = cxf(x)$ do not satisfy the conditions. If $a$, $b$ are both positive or negative then MVT leads to contradiction since there is a local maximum or minimum where $f(x)\neq 0$. If $0\in [a,b]$, we recursively take a derivative of $f(x)$: each derivative must be zero at the endpoints. Then by applying MVT and the derivative we find a dense set where all derivatives are zero, which is impossible because $f$ cannot be identically zero.

\subsection*{Problem 16}
(a) We separate the integral: \[s\int_1^{\infty}\frac{\lfloor x\rfloor}{x^{s+1}}\,dx=s\sum_{n=1}^{\infty}\int_n^{n+1}\frac{\lfloor x\rfloor}{x^{s+1}}\,dx = s\sum_{n=1}^{\infty}\int_n^{n+1}\frac{n}{x^{s+1}}\,dx \]
\[= \sum_{n=1}^{\infty}\left(\frac{n}{n^s}-\frac{n}{(n+1)^s}\right)\ \sum_{n=1}^{\infty}\left(\frac{n}{n^s}-\frac{n+1}{(n+1)^s}+\frac{1}{(n+1)^s}\right)\]
From here we cancel the adjacent terms in a telescoping series to achieve \[\sum_{n=1}^{\infty}\frac{1}{n^s}\]

(b) \[\frac{s}{s-1}-s\int_1^\infty \frac{x-\lfloor x\rfloor}{x^{s+1}}\,dx=\frac{s}{s-1}-s\int_1^\infty\frac{1}{x^s}\,dx+\int_1^\infty\frac{\lfloor x \rfloor}{x^{s+1}}\,dx=\int_1^\infty\frac{\lfloor x \rfloor}{x^{s+1}}\,dx\]

To prove convergence we bound the integral: $0\leq x-\lfloor x\rfloor\leq 1$, implying \[0\leq\frac{x-\lfloor x\rfloor}{x^{s+1}}\leq \frac{1}{x^{s+1}}\] for $x\geq 1$ and $s>0$. Then since $\int_1^\infty \frac{1}{x^{s+1}}$ converges our integral must also converge.

\subsection*{Problem 17}
As stated by the hint we may take the following lemma:

\medskip \noindent \textbf{Lemma:} For each $P=\{x_0,\dots,x_n\}$ some partition of $[a,b]$ there exists $t_1,\dots t_n$ with $x_{j-1}<t_j<x_j$ where \[\sum_{i=1}^n\alpha(x_i)g(t_i)\Delta x_i=G(b)\alpha(b)-G(a)\alpha(a)-\sum_{i=1}^nG(x_{i-1})\Delta \alpha_i\]
We take $t_i$ where $g(t_i)\Delta x_i = G(x_i)-G(x_{i-1})$ which exists via the Mean Value Theorem. From here \[\sum_{i=1}^n\alpha(x_i)g(t_i)\Delta x_i= \sum_{i=1}^n\alpha(x_i)G(t_i)- \sum_{i=1}^n\alpha(x_i)G(t_{i-1})\]
\[= \sum_{i=1}^n\alpha(x_i)G(t_i)-  \sum_{i=1}^n\alpha(x_{i-1})G(t_{i-1})-\sum_{i=1}^n\alpha(x_i)G(t_{i-1})+ \sum_{i=1}^n\alpha(x_{i-1})G(t_{i-1})\]
and then via telescoping series this evaluates to 
\[G(b)\alpha(b)-G(a)\alpha(a)-\sum_{i=1}^nG(x_{i-1})\Delta \alpha_i.\]

\medskip \noindent We note that if we take a fine enough partition the right side of this equation converges to the right side of our desired equation. As such, we aim to prove that for each $\epsilon>0$ there exists a partition such that $d(\sum_{i=1}^n\alpha(x_i)g(t_i)\Delta x_i,\sum_{i=1}^n\alpha(x_i)g(x_i)\Delta x_i) < \epsilon$. In this case the convergence of the second expression to the desired integral yields our final result.

\medskip \noindent Since $\alpha$ is monotonically increasing it is bounded by $\alpha(b)$ and $\alpha(a)$: take $M=\max(|\alpha(b)|,|\alpha(a)|)$. Now take $\epsilon_0 = \frac{\epsilon}{(b-a)M}$. Since $g$ is uniformly continuous we select $\delta$ such that no two points closer together than $\delta$ are mapped further than $\epsilon_0$ away. Take $P$ such that the distance between two break points is less than $\delta$. Then:
\[\left|\sum_{i=1}^n\alpha(x_i)g(t_i)\Delta x_i-\sum_{i=1}^n\alpha(x_i)g(x_i)\Delta x_i\right| = \left|\sum_{i=1}^n\alpha(x_i)(g(t_i)-g(x_i))\Delta x_i\right|\]\[\leq  \left|\sum_{i=1}^n\alpha(x_i)\epsilon_0\Delta x_i\right|\leq  \left|\sum_{i=1}^nM\epsilon_0\Delta x_i\right|=\epsilon\] by triangle inequality, and since decreasing $\delta$ brings the lower and upper sums within $\epsilon$ of each other, we are done.

\end{document}

