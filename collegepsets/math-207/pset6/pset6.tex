\documentclass{amsart}
\usepackage{../../../lucas}
\usepackage{amsmath, amssymb}
\usepackage{graphicx}

\title{Problem Set 6}
\author{Lucas\ Chen}
\date{\today}
\begin{document}

\maketitle

Problems: 4.27, 4.29, 4.41, 5.1, 5.2, 5.7, 5.8, 5.17, 5.20


















\subsection*{Problem 4.27} Suppose that \( f : M \rightarrow M \) and for all 
\( x, y \in M \), if \( x \neq y \) then \( d(fx, fy) < d(x, y) \). Such an \( f \) is a weak contraction.

(a) Is a weak contraction a contraction? (Proof or counterexample.)

\medskip \noindent No. Consider $f(x)=-\arctan{x} + x$. The derivative of this function
is $\frac{x^2}{1+x^2}$ which is bounded between $0$ and $1$. Thus the function is a weak contraction
by mean value theorem. However, the function is not a contraction 

\bigskip

(b) If \( M \) is compact, is a weak contraction a contraction? (Proof or counterexample.)

\medskip \noindent No. Note $f(x) = x^2$ on $[0, 1]$. Consider $d(a, 1)$ and $d(f(a), f(1))$: 

(c) If \( M \) is compact, prove that a weak contraction has a unique fixed-point.

\medskip \noindent We prove there is a fixed point. Take $d$ the infinum of $d(x, f(x))$ for each $x\in M$. Then $\exists$
a sequence $(x_n)$ such that $d(x_n, f(x_n))-d<1/n$. Since $M$ is compact we take a convergent
subsequence $(x_{n_k})$ with limit $x$. We know $\exists$ $N$ such that $d(x_m, f(x_m)) < d+\epsilon/3$ for each
$m>N$ and we can take $M$ such that $d(f(x_{n_k}), f(x))\leq d(x_{n_k}, x)<\epsilon/3$ for $k>M$. 
Then pick $n_k>k>\max(N, M)$. Then $d(x, f(x))\leq d(f(x_{n_k}), f(x))+ d(f(x_{n_k}), x_{n_k})+d(x_{n_k}, x)<\epsilon+d$
for each $\epsilon$: by $\epsilon$-principle we have $d(f(x), x)\leq d$ and $d(f(x), x)=d$ by definition.
If $d>0$, $f(x)\neq x$ implies $d(f(f(x)), f(x))<d(f(x), x)=d$ which is a contradiction. Thus $d=0$ and $x$
is the fixed point.

\newpage

\subsection*{Problem 4.29} Give an example to show that the fixed-point in Brouwer's Theorem need not be unique.

\medskip \noindent $f(x)=x$ on $[-1, 1]$ is a function on $B^1$ and each point in the graph of the function
is a fixed-point.

\newpage

\subsection*{Problem 4.41} (a) Give an example of a function 
\( f : [0,1] \times [0,1] \rightarrow \mathbb{R} \) such that for each fixed 
\( x \), the function \( y \mapsto f(x,y) \) is a continuous function of \( y \), 
and for each fixed \( y \), the function \( x \mapsto f(x,y) \) is a continuous function 
of \( x \), but \( f \) is not continuous.

\medskip \noindent $f(x)= \frac{xy}{x^2+y^2}$. 

\bigskip

(b) Suppose in addition that the set of functions
\[
\mathcal{E} = \{ x \mapsto f(x,y) : y \in [0,1] \}
\]
is equicontinuous. Prove that \( f \) is continuous.

\medskip \noindent Take some $x_1, y_1\in[0,1]$. We prove that for each $\epsilon$ $\exists$
a $\delta$ such that $d((x,y), (x_1, y_1))<\delta\implies d(f(x,y), f(x_1, y_1))<\epsilon$. 
For a given $\epsilon$ take $\delta_y$ such that $d(y, y_1)<\delta_y$ implies $d(f(x_1, y), f(x_1, y_1))<\epsilon/2$.
Then take $\delta_x$ such that $d(x, x_1)<\delta_x$ implies $d(f(x, y), f(x_1, y))<\epsilon/2$ for all $y$: this exists
due to the equicontinuity condition. Then take $\delta = \min(\delta_x, \delta_y)$. We have if
$d((x_1, y_1), (x, y))<\delta$ then $d(x_1, x), d(y_1, y)<\delta$ (using Euclidean, taxicab, and maximum metrics) and 
$d(f(x_1, y_1), f(x, y))\leq d(f(x, y), f(x_1, y))+d(f(x_1, y), f(x_1, y_1))<\epsilon$ by triangle
inequality. Thus continuity holds.

\newpage

\subsection*{Problem 5.1} Let \( T : V \rightarrow W \) be a linear transformation, 
and let \( p \in V \) be given. Prove that the following are equivalent.

(a) \( T \) is continuous at the origin.

(b) \( T \) is continuous at \( p \).

(c) \( T \) is continuous at at least one point of \( V \).

\medskip \noindent We prove (a) implies (b) first. Given $\epsilon$ we pick $\delta$ such that $||v-0||=||v||<\delta$ implies $||Tv-T(0)||=||Tv||<\epsilon$. Then
consider $x_1$ such that $d(p, x_1)<\delta$: since $V$ is a vector space we have 
$d(p, x_1)=||p-x_1||=||(p-x_1)-0||=d(p-x_1, 0)$. Then $||T(p-x_1)||=||T(p)-T(x_1)||<\epsilon$.

\medskip \noindent (b)$\implies$(c) is obvious since $p\in V$.

\medskip \noindent We prove (c) implies (a). Take this point $x$. Then given $\epsilon$ we have $\delta$ where
$||x-x_1||<\delta$ implies $||T(x_1)-T(x)||<\epsilon$. Take $||v||<\delta$. Then $||x-(x-v)||=||v||<\delta$
implies $||T(x)-T(x-v)||=||T(x)-T(x)+T(v)||=||T(v)||<\epsilon$, implying continuity at the origin.

\newpage

\subsection*{Problem 5.2} Let \( \mathcal{L} \) be the vector space of continuous 
linear transformations from a normed space \( V \) to a normed space \( W \). Show that 
the operator norm makes \( \mathcal{L} \) a normed space.

\medskip \noindent We prove the three properties. Take $T, T_1, T_2\in\mathcal{L}$ and the operator norm
$||\cdot||$. Since $|Tv|_W$ and $|v|_V$ are nonnegative so is $||T||$. If $||T||$ is $0$ then 
\[\frac{|Tv|_W}{|v|_V}\leq 0\implies |T_v|_W=T_v=0\] for all $v$, so $T=0$. If $T=0$ then clearly 
$\frac{|Tv|_W}{|v|_V}=0$ for all $v\neq 0$: then $||T||=0$. 

\medskip \noindent We now prove the triangle inequality. We have 
\[||T_1||\geq\frac{|T_1v|_W}{|v|_V}, ||T_2||\geq\frac{|T_2v|_W}{|v|_V}\] for each $v\neq 0$ in $V$.
Then \[||T_1||+||T_2||\geq\frac{|T_1v|_W+|T_2v|_W}{|v|_V}\geq\frac{|(T_1+T_2)v|_W}{|v|_V}.\]
Since $||T_1+T_2||=\sup\{\frac{|(T_1+T_2)v|_W}{|v|_V}\}$ we have $||T_1||+||T_2||\geq||T_1+T_2||$.

\medskip \noindent We prove the third property. Assume $\lambda$ a positive scalar.
Then $||\lambda T||$ = $\sup\{\frac{\lambda|T|_W}{|v|_V}\}$ We claim $||\lambda T||=\lambda||T||$:
$||T||\geq\frac{|T|_W}{|v|_V}$ implies $\lambda||T||\geq\frac{\lambda|T|_W}{|v|_V}$, and if $\exists s$ where $s\geq\frac{\lambda|T|_W}{|v|_V}$
with $s<\lambda||T||$ then $s/\lambda\geq\frac{|T|_W}{|v|_V}$ implies $s/\lambda$ is an lower upper bound for $\frac{|T|_W}{|v|_V}$ which is impossible.

\newpage

\subsection*{Problem 5.7} Two norms \( \| \cdot \|_1 \) and \( \| \cdot \|_2 \) on a vector 
space are comparable if there are positive constants \( c \) and \( C \) such that for all 
nonzero vectors in \( V \) we have
\[
c \leq \frac{\| v \|_1}{\| v \|_2} \leq C.
\]

(a) Prove that comparability is an equivalence relation on norms.

\medskip \noindent Reflexivity: $\frac{||v||_1}{||v||_1}=1$ which is between $1/2$ and $3/2$.

\medskip \noindent Symmetry:  $\frac{\| v \|_1}{\| v \|_2} \leq C\implies\frac{\| v \|_2}{\| v \|_1} \geq 1/C$, 
$\frac{\| v \|_1}{\| v \|_2} \geq c\implies\frac{\| v \|_2}{\| v \|_1} \leq 1/c$ and since $c, C>0$, we know $1/c, 1/C>0$.

\medskip \noindent Transitivity: $c_1 \leq \frac{\| v \|_1}{\| v \|_2} \leq C_2$ and $c_2 \leq \frac{\| v \|_2}{\| v \|_3} \leq C_2$
implies $c_1c_2 \leq \frac{\| v \|_1}{\| v \|_3} \leq C_1C_2$ which are both greater than $0$ since each of the factors are.

\bigskip

(b) Prove that any two norms on a finite-dimensional vector space are comparable. [Hint: Use Theorem 3.]

\medskip \noindent Let $V$ be of dimension $n$ and take norms $|\cdot|_1$ and $|\cdot|_2$. Take $T$ an isomorphism (bijective operator) from
$\mathbb{R}^n$ to $V$: $T$ is a homeomorphism by theorem 3. For any $x$ in $\mathbb{R}^n$, we have 
$c_1, c_2$ where $\frac{|Tx|_1}{c_1}\leq |x|\leq c_1|Tx|_1$, and $\frac{|Tx|_2}{c_2}\leq |x|\leq c_2|Tx|_2$. Then 
$\frac{1}{c_1c_2}\leq \frac{|T|_1}{|T|_2}\leq c_1c_2$ and we have the two norms comparable.

(c) Consider the norms
\[
\| f \|_{L_1} = \int_0^1 | f(t) | \, dt \quad \text{and} \quad \| f \|_{C^0} = \max \{ | f(t) | : t \in [0,1] \},
\]
defined on the infinite-dimensional vector space \( C^0 \) of continuous functions 
\( f : [0,1] \rightarrow \mathbb{R} \). Show that the norms are not comparable by 
finding functions \( f \in C^0 \) whose integral norm is small but whose \( C^0 \) norm is 1.


\medskip \noindent Take the set $\{f_n=t^n: n\in\mathbb{N}\}$ on $[0,1]$. We have
$||f_n||_{L_1}=1/(n+1)$ and $||f_n||_{C^0}=1$. Then the infinum of $\frac{||f_n||_{L_1}}{||f_n||_{C^0}}$ is
$0$ and the norms are not comparable.

\newpage

\subsection*{Problem 5.8} Let \( \| \cdot \| = \| \cdot \|_{C^0} \) be the supremum norm on 
\( C^0 \) as in the previous exercise. Define an integral transformation \( T : C^0 \rightarrow C^0 \) by
\[
T : f \mapsto \int_0^x f(t) \, dt.
\]

(a) Show that \( T \) is linear, continuous, and find its norm.

\medskip \noindent $T(\lambda f) = \int_0^x \lambda f(t)\,dt=\lambda\int_0^xf(t)\,dt=\lambda Tf$
and $T(f_1+f_2)=\int_0^xf_1+f_2\,dt=\int_0^xf_1\,dt+\int_0^xf_2\,dt=Tf_1+Tf_2$. Thus $T$ is linear.

\medskip \noindent $T$ is continuous since it is a constant away from the indefinite integral of $f(x)$
and the uniform convergence (convergence under sup norm) of $f$s yields the uniform convergence of indefinite
integrals of $f$ and therefore the uniform convergence of $T$ (since the indefinite integrals of $f$ at $0$ 
must also converge by uniform convergence; the sum of convergent sequences is convergent).

\bigskip

(b) Let \( f_n(t) = \cos(nt) \), \( n = 1, 2, \dots \). What is \( T(f_n) \)?

\medskip \noindent $\sin(nx)/n$. 

(c) Is the set of functions \( K = \{ f_n : n \in \mathbb{N} \} \) closed? Bounded? Compact?

\medskip \noindent It's bounded for obvious reasons. It is closed but not compact. (Out of time to justify closed, but 
it's not equicontinuous and so isn't compact.)

(d) Is \( T(K) \) compact? How about its closure?

\medskip \noindent $T(K)$ is not compact, since its uniform limit converges to 0, which is not in the set.
Its closure is compact, however, since the set $\sin(nx)/n$ is equicontinuous (its derivative is bounded between
-1 and 1) and so the closure is closed, equicontinuous, and trivially bounded.

\newpage

\subsection*{Problem 5.17} Let \( f : U \rightarrow \mathbb{R}^m \) be differentiable, 
\( [p,q] \subset U \subset \mathbb{R}^n \), and ask whether the direct analog of the 
one-dimensional Mean Value Theorem is true: Does there exist a point \( \theta \in [p,q] \) such that
\[
f(q) - f(p) = (Df)_{\theta}(q - p)?
\]

(a) Take \( n = 1 \), \( m = 2 \), and examine the function
\[
f(t) = (\cos t, \sin t)
\]
for \( \pi \leq t \leq 2\pi \). Take \( p = \pi \) and \( q = 2\pi \). Show that 
there is no \( \theta \in [p,q] \) which satisfies (28).

\medskip \noindent We have $f(q)-f(p)=(2,0)$, $Df=(-\sin t, \cos t)$, and $q-p=\pi$.
If $\cos t\neq 0$ then $\theta=0$ leads to a contradiction, and if $\cos t=0$ we have 
$-\sin t = 1$ does not satisfy (28).

(b) Assume that the set of derivatives
\[
\{ (Df)_x \in \mathcal{L}(\mathbb{R}^n, \mathbb{R}^m) : x \in [p,q] \}
\]
is convex. Prove there exists \( \theta \in [p,q] \) which satisfies (28). 
[Hint: Google "support plane."]

(c) How does (b) imply the one-dimensional Mean Value Theorem?

\medskip \noindent We have in $\mathbb{R}$ that each interval is connected and each connected
set is convex, so we must prove that the set of derivatives $\{f'(x): x\in[p,q]\}$ is an interval. 
The MVT requres that $f$ is $C^1$, so we have $f'$ continuous: since $[p,q]$ is connected and compact
(an interval), $f'([p,q])$ is connected and compact and therefore an interval.

\newpage

\subsection*{Problem 5.20} Assume that \( U \) is a connected open subset 
of \( \mathbb{R}^n \) and \( f : U \rightarrow \mathbb{R}^m \) is differentiable 
everywhere on \( U \). If \( (Df)_p = 0 \) for all \( p \in U \), show that \( f \) is constant.

\medskip \noindent By the Mean Value Theorem (Theorem 11 of Chapter 5), $|f(q)-f(p)|\leq 0|q-p|$ for
all $q, p$ with the segment between them in $U$. Since $U$ is a connected open subset of $\mathbb{R}^n$
it is path connected and therefore any two points $x_1$ and $x_2$ of $U$ can be connected via segments 
(take a finite closed neighborhood of a point on the path and select a further point on the sphere, on the path between $x_1$ and $x_2$,
since the path is compact these neighborhoods cannot converge to a point along the middle of the path)
and
the function must be constant. 

\newpage


\end{document}

