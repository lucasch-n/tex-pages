\documentclass{amsart}
\usepackage{../../../lucas}
\usepackage{amsmath, amssymb}
\usepackage{graphicx}

\title{Problem Set 1}
\author{Lucas Chen}
\date{\today}
\begin{document}

\maketitle


\subsection*{Lemma 1} (General lemma I will use): For a cut $x=A|B$ if $a'\in\mathbb{Q}$ and $a'<a$ for $a\in A$ then $a'\in A$:

\medskip
\noindent \textbf{Proof:} Since $a'\in\mathbb{Q}$ and $A\cup B=\mathbb{Q}$ and $A\cap B=\emptyset$ then if $a'\notin B$ it must be in
$A$. Then $a'\in A$ since if $a'\in B$ by definition $a'>a$, which is impossible.

\subsection*{Problem 9}
\noindent Let $x = A|B$, $x' = A'|B'$ be cuts in $\mathbb{Q}$. We defined $$x+x'=(A+A') \text{ }| \text{ rest of }\mathbb{Q}$$
\indent (a) Show that although $B + B'$ is disjoint from $A + A'$, 
it may happen in degenerate cases that $\mathbb{Q}$ is not the union of $A + A'$ and $B + B'$.

\medskip
\noindent \textbf{Sol:} For a degenerate case, we consider $x$, $x'$ for which $A = \{x: x<0$ or $x^2<2\}$, and 
$A' = \{x' : (x'-1)^2>2$ and $x'<0\}$. 

\medskip
\noindent Intuitively, these cuts represent real numbers $\sqrt{2}$ and $1-\sqrt{2}$, 
but since they are irrational their corresponding $B$-sets do not contain their greatest lower bounds. $B+B'$
therefore also does not include $1$, which means $1$ is excluded from the sets $A+A'$ and $B+B'$. This means $(A+A')$ $|$ $(B+B')$
is by definition not a cut.

\medskip
\noindent Rigorously, we prove that $\nexists a\in A, a'\in A'$ where $a+a'=1$, and $\nexists b\in B, b'\in B'$ where $b+b'=1$, 
so $1\notin A+A' \cup B+B'$. We note that the conditions for $B$ and $B'$ are $B= \{y: y\geq 0$ and $y^2\geq 2\}$ 
$B'= \{y' : (y'-1)^2\leq2$ or $y'\geq0\}$.

\medskip
\noindent We first assume the negative of our desired result for the $A$-sets. Then given $a$ s.t. $a<0$ or $a^2<2$, we plug in
$a'=1-a$ to the condition for $A'$: $$a^2>2 \text{ and } a>1$$ which yields us a contradiction since $a\in A$. Now we test the 
corresponding hypothesis for $B$, $B'$: Given $b\in B'$ we plug in $1-b$ to $B$'s conditions: $b\leq 1$ and $(1-b)^2=(b-1)^2\geq 2$
yields a contradiction. Then $A+A'\cup B+B'$ excludes $1$.

\bigskip

\indent (b) Infer that the definition of $x + x'$ as $(A + A')\text{ }| \text{ }(B + B')$ would be incorrect.

\medskip
\noindent \textbf{Sol:} As stated above $A+A'\cup B+B'$ excludes $1$ and $\neq \mathbb{Q}$: thus $(A + A')\text{ }| \text{ }(B + B')$
is not a cut.

\bigskip

\indent (c) Why did we not define $x\cdot x' =(A\cdot A')\text{ } |\text{ rest of }\mathbb{Q}$?

\medskip
\noindent \textbf{Sol:} Given two positive cuts $A$ $|$ $B$ and $A'$ $|$ $B'$ $A\cdot A'$ includes all of $\mathbb{Q}^+$ since $\mathbb{Q}^- \subset A$
and $\mathbb{Q}^- \subset A'$. Thus the cut $(A\cdot A')\text{ } |\text{ rest of }\mathbb{Q}$ does not make sense because every product of positive
numbers would correspond to $\infty$ or be undefined.




\newpage
\subsection*{Problem 10}
\noindent Prove that for each cut $x$ we have $x + (-x) = 0\text{*}$

\medskip
\noindent \textbf{Sol:} From the definition of the cut $-x$ we have $x + (-x) = A$ $|$ $B$ where $A = \{x : x = a+b, a\in A', -b \in B', -b$ 
not a lower bound of $B'\}$, for $x = A'$ $|$ $B'$. We note that by definition of a cut $-b > a$, so $a+b<0$ and $A$ is bounded above by $0$.

\medskip
\noindent We now prove that $\forall$ $c<0$, $c = a+b \in A$ for some $a\in A'$, $-b\in B'$. 
To do this, we pick an arbitrary $a'$ in $A'$ and an arbitrary $b'$ in $B'$, since neither can be the empty set. Given $c$ we can find the 
multiple of $c$ by the smallest natural $d$ where $d\in\mathbb{N}$, and $cd < a'-b'$, since every rational is in an interval of integers — 
$a'-cd>b'\implies a'-cd\in B'$. 

\medskip
\noindent From here we note that $a' - cn$ for $n\in \mathbb{N} \cap [0, d]$ is a rational and therefore $a'-cn\in A'$ or 
$a'-cn\in B'$. As such $\exists n$ s.t. $a'-cn\in A'$ and $a'-c(n+1)\in B'$, since $a'\in A'$ and $a'-cd\in B'$. Thus we set $a=a'-cn$ and 
$-b=a'-c(n+1)\implies a+b=c$. 

\medskip
\noindent In the case that $a'-c(n+1)$ is the l.u.b. of $A'$, we may add a rational $f$ less than $c$ to both values for satisfactory
elements of $A'$ and $B'$ — since $a'-c(n+1)$ is the l.u.b., $a'-cn +f<a'-c(n+1)\implies a'-cn+f\notin B'$ but in $A'$. Then in all cases $\exists$ 
elements $a$ and $b$ that satisfy $a+b=c$ for $c<0$ and the cut is the zero cut. 

\subsection*{Lemma 2}  Multiplication by a positive real preserves inequalities and equalities.

\medskip
\noindent \textbf{Proof:} Given $a,b\in \mathbb{R}$ with $a<b$ and $r\in\mathbb{R}^+$ we aim to prove $ar<br$. For $a<b$ we have $a+(-a)<b+(-a)$
by translation or $0<b-a$. Then $b-a$ is a positive real number. Then we have $r(b-a)>0$ as a property of multiplication, and we use the 
distributive property for $br-ar>0$. By translation $br>ar$. 

Equality is proven since the lower sets of the equal cuts are the same and so the product cuts have the same definition. Equality-inclusive
inequalities are consistent by considering the case of inequality and the case of equality.




\newpage
\subsection*{Problem 11}
\noindent A multiplicative inverse of a nonzero cut $x = A\text{ } | \text{ }B$ is a cut $y = C\text{ } | \text{ }D$ such that
 $x\cdot y=1$*.

\indent (a) If $x>0$*, what are $C$ and $D$?

\medskip
\noindent \textbf{Sol:} We posit that $C = \{x\in\mathbb{Q}: x\leq 0 $ or $\frac{1}{x}\in B$ but not g.l.b. of $B\}$.
(Note: we define g.l.b. later but
here it only matters if it is the smallest element of $B$, so we use it in place of that term, i.e. I do not have the time to replace "g.l.b." with smallest
element everywhere.)

\medskip
\noindent \textbf{Proof:} $x>0$ and $1>0\implies y>0$, so we have for $x=A|B$ and $y=C|D$ $xy= E'|F'$ where 
$$E' = \{r\in\mathbb{Q}:r\leq0 \text{ or } \exists a\in A \text{ and } \exists c\in C \text{ such that } a>0, c>0, \text{ and } r=ac\}$$
We prove for $1= E|F$ that $E\subset E'$ and $E'\subset E$. First we prove that $E'$ is bounded above at 1, which implies it is a 
subset of $E$. Assume $\exists r\geq 1$ where $r=ac$ for $a>0, c>0, a\in A, c\in C$. We have $a\in A$, $\frac{1}{c}\in B$, so $a<\frac{1}{c}$:
however, $r=ac\implies a=\frac{r}{c}$ for $r\geq 1$ meaning $a\geq c$ which is a contradiction.

\medskip
\noindent Now we prove that $r\in E'$ $\forall$ $r<1, r\in\mathbb{Q}$. If $r\leq 0$, $r\in E'$ by definition. We aim to, for $r\in (0,1)$,
find $a\in A$ and $b\in B (b\neq $ l.u.b $B)$ such that $\frac{a}{b}=r$. 

\medskip
\noindent As in Problem 10 we select arbitrary $a'\in A$ and $b'\in B$. Since
$a'$, $b'$ are positive rationals with $b'>a'$ and $1/r > 1$, we have $\frac{b'}{a'}< (1/r)^d$ for some $d\in\mathbb{N}$. Then we have 
$a'(1/r)^d\in B$ and $\exists$ $n\in\mathbb{N}$ s.t. $a'(1/r)^n\in A$ and $a'(1/r)^{n+1}\in B$. Setting $a=a'(1/r)^n$ and $b= a'(1/r)^{n+1}$
yields us $a\cdot\frac{1}{b}=r$. 

\medskip
\noindent If incidentally $b$ is the lower bound of $B$ we pick $s\in (1, 1/r)$ and choose $a=a'(1/r)^ns$ and 
$b=a'(1/r)^{n+1}s$ where $a\in A$ since $s<1/r$. Thus, $E\subset E'$ and we have $E=E'$ and $y = \frac{1}{x}$. 

\bigskip

\indent (b) If $x<0$*, what are they?

\medskip
\noindent \textbf{Sol:} If $x<0$ then $y<0$, and $x\cdot y$ is defined by $(-x) \cdot (-y)$. We write $-x = A'|B'$ and $-y = C'|D'$. We have 
$$A' = \{a'\in\mathbb{Q} : a' = -b \text{ for some }b \in B, b \text{ not a lower bound}\}$$
Then $$C'= \{c'\in\mathbb{Q} : c'\leq 0\text{ or }\frac{1}{c'}\in B', \frac{1}{c'} \text{ not a lower bound.}\}$$ 
Two cases: If $B$ includes a lower bound $x$, then $-x\in B'$ since it is excluded from $A'$. Note here that $-x$ becomes the lower bound
of $B'$ since $x$ is an upper bound for $A$ (whose negatives are excluded from $A'$). Then $-\frac{1}{x}$ is excluded from $C'$. If $B$ does not
include a lower bound then naturally $C'$ does not include a function of the lower bound of $B'$ since it does not exist. 

\medskip
\noindent From here, we combine the two sets: $$B'=\{b'\in\mathbb{Q}: b'=-a \text{ for some }a \in A \text{ or }a \text{ lower bound of }B\}$$ 
$$C'=\{c'\in\mathbb{Q}: c'\leq 0\text{ or }-\frac{1}{c'}\in A\}$$
We can then take the negative of this cut: 
$$D'=\{d'\in\mathbb{Q}: d'>0\text{ and }-\frac{1}{d'}\in B\}$$
$$C=\{c\in\mathbb{Q}: -c=d'>0\text{ and }\frac{1}{c}\in B \text{ but not g.l.b. of} B\}$$

(note: the g.l.b. of $D'$ is $-\frac{1}{b}$ for $b$ the g.l.b. of $B$)
and we have $$C =\{c\in\mathbb{Q}: c<0\text{ and }\frac{1}{c}\in B \text{ but not g.l.b of } B\}$$


\bigskip

\indent (c) Prove that $x$ uniquely determines $y$.

\medskip
\noindent \textbf{Sol:} We must consider the cases of positive and negative $x$. For positive $x$ assume there exists $y\neq y'$: we will prove the contrapositive,
that $xy \neq xy'$ and thus $\exists$ a unique $y$ for $xy = 1$. 

\medskip
\noindent We have $y = C|D$ and $y' = C'|D'$: we will assume WLOG that $y>y'$ (trichotomy) so that $\exists s\in C, s\notin C'$. Then we write
$xy = E|F$ with $E=\{r\in \mathbb{Q}: r\leq 0 $ or $r=ac$ for $a\in A\cap\mathbb{Q}^+$ and $c\in C\cap\mathbb{Q}^+\}$ and $xy'= E'|F'$ with $E'
= \{r' :$ defined similarly as above.$\}$ 

\medskip
\noindent We plug $s$ into $E$ to get $r = as$: since $a\in\mathbb{Q}^+$ $r>ac'$ $\forall$ $c'\in C'$ so $\exists$ $r\in E$, $r\notin E'$. Then 
$E\neq E'$.

\medskip
\noindent For the negative case, we define $y$ and $y'$ similarly as above and assume $y>y'$. Then since $\exists s\in c, s\notin C'$, $s\in D'
, s\notin D$. (We note that we can pick $s$ not the lower bound of $D'$ since if only the lower bound is in $D'$ but not $D$, then $s$ is necessarily
the least upper bound of $C$ and in $C$, which is impossible). Then for $-y= C^{-} | D^{-}$ and  $-y'=C^{-'} | D^{-'}$ we have $-s\in C^{-'}$, $-s\notin
C^{-}$. From here, we compare $C^{-'}$ and $C^{-}$ to $-x$ as in the positive case to achieve $E\neq E'$. 





\newpage
\subsection*{Problem 13}
\noindent Let $b = $ l.u.b $S$, where $S$ is a bounded nonempty subset of $\mathbb{R}$.

\indent (a) Given $\epsilon > 0$ show that there exists an $s\in S$ with $$b-\epsilon\leq s\leq b.$$

\medskip
\noindent \textbf{Sol:} Since $b$ is the l.u.b., $\forall$ $s\in S$ $b\geq s$. Thus if $\nexists s$ for $b-\epsilon\leq s\leq b$ then $s<b-\epsilon$ $\forall$
 $s\in S$. Then $b-\epsilon$ is an upper bound of $S$, and since $b-\epsilon<b$, this is a contradiction.

\bigskip

\indent (b) Can $s\in S$ always be found so that $b-\epsilon<s<b$?

\medskip
\noindent \textbf{Sol:} No, because of the counterexample $S = {b}$, which $b$ is an upper bound for. 

\bigskip

\indent (c) If $x=A$ $|$ $B$ is a cut in $\mathbb{Q}$, show that $x=$ l.u.b.$A$.

\medskip
\noindent \textbf{Sol:} For $a\in A$ we have $a^*\in\mathbb{R}$ with $a^* = A'|B'$ $A'=\{a'<a\}$ the rational cut in $\mathbb{R}$ corresponding to $a$. Then
we have $A'\subset A$ since $\forall$ $a'\in A'$ we have $a'<a\in A$: if $a'\notin A$ then $a\in B$ and $a'>a$, contradiction. Then $x$ is an
upper bound of $A^* = \{a^*\}$. To prove $x$ is the least upper bound assume $\exists$ an upper bound y of $A^*$. Then for $a$ in $A$ we have 
$a^*\leq y$. If $a^* = y$ then $y$ is a rational cut and therefore in $A$, but since $a^*<y$ $\forall$ $a\in A$ ($y$ is an upper bound
of rational set $A$) this contradicts the definition of the cut $x$. Thus $a^*<y$ implies $x\leq y$, and $x$ is the l.u.b..




\newpage
\subsection*{Problem 14}
\noindent Prove that $\sqrt{2}\in\mathbb{R}$ by showing that $x\cdot x=2$ where $x=A|B$ is the cut
in $\mathbb{Q}$ with $A=\{r \in \mathbb{Q}: r\leq 0 $ or $r^2<2\}$.

\medskip
\noindent \textbf{Sol:} We note that $x$ is a positive cut, so $x^2=y=E|F$ where $E= \{r\in\mathbb{Q}: r\leq 0$ or $r=ab$ for $a\in A\cap\mathbb{Q}^+$
and $b\in A\cap\mathbb{Q}^+\}$. We also note that if an element $r$ of one set $\leq 0$ it must be in the other by definition.

\subsection*{Lemma:} Given $0<r<2$ $\exists a\in\mathbb{Q}$ s.t. $a^2\in[r, 2)$.

\noindent \textbf{Proof:} By problem 13 we have that $x=\sup A^*$ with $A^* =\{a^*\in\mathbb{R}: a^*$ for a in $A\}$. Then we define $\epsilon=
\frac{x^2-r}{2x}$. Then $\exists a\in A$ s.t. $x-\epsilon\leq a^*\leq x$. We note that $x-\epsilon = \frac{x^2+r}{2x}>0$. Then 
$$x-\epsilon\leq a^*\implies x^2-2x\epsilon\leq (x-\epsilon)^2\leq a^{*2} < 2$$ since $a > 0$ and $a\in A$. (Note $(x-\epsilon)^2\leq a^*(x-\epsilon)\leq a^{*2}$
by Lemma 2, and$(x-\epsilon)^2\leq a^{*2}$ by transitivity.) 
This yields us $r\leq a^2<2$ since an inequality in $\mathbb{R}$ of rational cuts implies the corresponding inequality in the rationals. 

From here, the proof follows: $r\leq a^2\implies \frac{r}{a}\leq a$, and $\frac{r}{a}>0\implies (\frac{r}{a})^2\leq a^2 <2$, and $\frac{r}{a}
\in A$. Thus we define $b=\frac{r}{a}$ and we have achieved $ab=r$ $\forall r>0, r\in E$. Then $2\leq x^2$. 

\medskip
\noindent For the opposite direction we assume $r\in E$ and we prove $r < 2$. Assume $r\geq 2$. Then $r=ab$ with $a,b>0\implies a^2b^2\geq 4$, 
but $a^2, b^2<2\implies a^2b^2<4$: contradiction. Thus $x^2\leq 2$ and $x\cdot x = 2$. 





\newpage
\subsection*{Problem 18}
\noindent Prove that real numbers correspond bijectively to decimal expansions not terminating in 
an infinite string of nines, as follows. The decimal expansion of $x\in\mathbb{R}$ is $N.x_1x_2\dots,$ 
where $N$ is the largest integer $\leq x$, $x_1$ is the largest integer $\leq 10(x-N)$, $x_2$ is the
largest integer $\leq 100(x-(N+x_1/10))$, and so on.

\indent (a) Show that each $x_k$ is a digit between 0 and 9.

\medskip
\noindent \textbf{Sol:} Describe $x_k$ the greatest integer $\leq 10^k(x-(N+\sum\limits_{j=1}^{k-1}x_j/10^j))$. Thus, 
$10^k(x-(N+\sum\limits_{j=1}^{k-1}x_j/10^j))-x_k<1$ and $$10^{k+1}(x-(N+\sum\limits_{j=1}^kx_j/10^j))<10$$ implies for $k\geq 2$ $x_k<10$ and
is an digit from $0-9$. For $x_1$, since $x-N<1$ and $10(x-N)<10$, $x_1$ is a digit.  

\bigskip

\indent (b) Show that for each $k$ there is an $l\geq k$ such that $x_l\neq 9$.

\medskip
\noindent \textbf{Sol:} Assume $\exists$ $k$ where $x_l = 9$ $\forall l > k$. Then $\frac{9}{10^l}\leq x-(N+\sum\limits_{j=1}^{l-1}x_j/10^j)$ 
$\forall l > k$ and $$x\geq N+\sum\limits^k_{j=1}\frac{x_j}{10^j}+\sum\limits^l_{m=k+1}\frac{9}{10^m}$$ 

\medskip
\noindent We aim to prove that $\forall$ $a\geq\sum\limits^l_{m=k+1}\frac{9}{10^m}$ $\forall$ $l>k$, $a\geq\frac{1}{10^k}$ For this, we aim 
to use $\epsilon$-principle. Define $y = \frac{1}{10^k}$. Then for any upper bound of our sequence $b$, and any given $\epsilon>0$, we can
take the least $c\in\mathbb{N}$ so that $10^c\geq1/\epsilon$. Then if $c>k$, $\epsilon\geq1/10^c\implies
\frac{1}{10^k}-\epsilon\leq\frac{1}{10^k}-\frac{1}{10^c}=\sum\limits^l_{m=k+1}\frac{9}{10^m}\leq y$ implies $\frac{1}{10^k}-\epsilon\leq y$
and $\frac{1}{10^k} \leq y$ by $\epsilon$-principle. Then $$x\geq N+\sum\limits^k_{j=1}\frac{x_j}{10^j}+\frac{1}{10^k}$$ 
$$x_k+1\leq 10^k(x-(N+\sum\limits_{j=1}^{k-1}\frac{x_j}{10^j}))$$ which contradicts the definition of $x_j$. By proof by 
contradiction, there must $\exists$ some $l\geq k$ $\forall k$ s.t. $x_l\neq 9$. 


\bigskip

\indent (c) Conversely, show that for each such expansion $N.x_1x_2\dots$ not terminating in an infinite
string of nines, the set $$\{N, N+\frac{x_1}{10}, N+\frac{x_1}{10}+\frac{x_2}{100}, \dots\}$$ is bounded 
and its least upper bound is a real number $x$ with decimal expansion $N.x_1x_2\dots$.

\noindent We prove first that $x$ is an upper bound of the given set. By the definition of each $x$-term, we have 
$$x_k\leq 10^k(x-(N+\sum\limits_{j=1}^{k-1}x_j/10^j))\implies x\geq N+\sum_{j=1}^{k}x_j/10^j$$ for each 
$j\in\mathbb{N}$. Thus $x$ is an upper bound by definition. 

\medskip
\noindent We then prove that $x$ is the least upper bound. Assume $\exists y<x$ with $y$ an upper bound of
our sequence. Then $\exists$ $r_1, r_2, r_1\neq r_2,$ By definition $r_1$, $r_2$ are upper bounds of the set.

\medskip
\noindent We have: for $s_k = N+\sum_{j=1}^{k}x_j/10^j$ $s_k<r_1$ $\forall$ $k\in\mathbb{N}$. Then 
$s_k + (r_2-r_1)< r_2$ $\forall k\in\mathbb{N}$. We consider the least $k$ such that $1/10^k<(r_2-r_1)$, which exists
since $s_k$ and $r_2-r_1$ are rationals: then $s_k + 1/10^k<s_k+(r_2-r_1)<r_2<x$. This yields us 
$$x_k+1< 10^k(x-(N+\sum^{k-1}_{j=1} \frac{x_j}{10^j}))$$ which contradicts the definition of $x_k$. Therefore, $\nexists$
$y<x$ a lower bound of our set $\{s_k\}$.

\medskip
\noindent Thus, the decimal expansion is well-defined, and the real number $x$ may be derived as the l.u.b. of $N, x_1,\dots$.
By definition this makes the decimal expansion invertible and a bijection.

\bigskip
\indent (d) Repeat the exercise with a general base in place of $10$.

\medskip
\noindent We copy the entire proof again and replace $10$ with $10_b$ for an arbitrary base $b$, and $9$ 
with $b-1$. 

\noindent Prove that real numbers correspond bijectively to decimal expansions not terminating in 
an infinite string of nines, as follows. The decimal expansion of $x\in\mathbb{R}$ is $N.x_1x_2\dots,$ 
where $N$ is the largest integer $\leq x$, $x_1$ is the largest integer $\leq 10(x-N)$, $x_2$ is the
largest integer $\leq 100(x-(N+x_1/10))$, and so on.

\indent (d-a) Show that each $x_k$ is a digit between 0 and $b-1$.

\medskip
\noindent \textbf{Sol:} (Note that $10_b = b_{10}$.) Describe $x_k$ the greatest integer $\leq 10_b^k(x-(N+\sum\limits_{j=1}^{k-1}x_j/10_b^j))$. Thus, 
$10_b^k(x-(N+\sum\limits_{j=1}^{k-1}x_j/10_b^j))-x_k<1$ and $$10_b^{k+1}(x-(N+\sum\limits_{j=1}^kx_j/10_b^j))<10_b$$ implies for $k\geq 2$ $x_k<10_b$ and
is an digit from $0-(b-1)$. For $x_1$, since $x-N<1$ and $10_b(x-N)<10_b$, $x_1$ is a digit.  

\bigskip

\indent (d-b) Show that for each $k$ there is an $l\geq k$ such that $x_l\neq b-1$.

\medskip
\noindent \textbf{Sol:} Assume $\exists$ $k$ where $x_l = b-1$ $\forall l > k$. Then $\frac{b-1}{10_b^l}\leq x-(N+\sum\limits_{j=1}^{l-1}x_j/10_b^j)$ 
$\forall l > k$ and $$x\geq N+\sum\limits^k_{j=1}\frac{x_j}{10_b^j}+\sum\limits^l_{m=k+1}\frac{b-1}{10_b^m}$$ 

\medskip
\noindent We aim to prove that $\forall$ $a\geq\sum\limits^l_{m=k+1}\frac{b-1}{10_b^m}$ $\forall$ $l>k$, $a\geq\frac{1}{10_b^k}$ For this, we aim 
to use $\epsilon$-principle. Define $y = \frac{1}{10_b^k}$. Then for any upper bound of our sequence $d$, and any given $\epsilon>0$, we can
take the least $c\in\mathbb{N}$ so that $10_b^c\geq1/\epsilon$. Then if $c>k$, $\epsilon\geq1/10_b^c\implies
\frac{1}{10_b^k}-\epsilon\leq\frac{1}{10_b^k}-\frac{1}{10_b^c}=\sum\limits^l_{m=k+1}\frac{b-1}{10^m}\leq y$ implies $\frac{1}{10_b^k}-\epsilon\leq y$
and $\frac{1}{10_b^k} \leq y$ by $\epsilon$-principle. Then $$x\geq N+\sum\limits^k_{j=1}\frac{x_j}{10_b^j}+\frac{1}{10_b^k}$$ 
$$x_k+1\leq 10_b^k(x-(N+\sum\limits_{j=1}^{k-1}\frac{x_j}{10_b^j}))$$ which contradicts the definition of $x_j$. By proof by 
contradiction, there must $\exists$ some $l\geq k$ $\forall k$ s.t. $x_l\neq b-1$. 


\bigskip

\indent (d-c) Conversely, show that for each such expansion $N.x_1x_2\dots$ not terminating in an infinite
string of $b-1$s, the set $$\{N, N+\frac{x_1}{10_b}, N+\frac{x_1}{10_b}+\frac{x_2}{100_b}, \dots\}$$ is bounded 
and its least upper bound is a real number $x$ with decimal expansion $N.x_1x_2\dots$.

\noindent We prove first that $x$ is an upper bound of the given set. By the definition of each $x$-term, we have 
$$x_k\leq 10_b^k(x-(N+\sum\limits_{j=1}^{k-1}x_j/10_b^j))\implies x\geq N+\sum_{j=1}^{k}x_j/10_b^j$$ for each 
$j\in\mathbb{N}$. Thus $x$ is an upper bound by definition. 

\medskip
\noindent We then prove that $x$ is the least upper bound. Assume $\exists y<x$ with $y$ an upper bound of
our sequence. Then $\exists$ $r_1, r_2, r_1\neq r_2,$ By definition $r_1$, $r_2$ are upper bounds of the set.

\medskip
\noindent We have: for $s_k = N+\sum_{j=1}^{k}x_j/10_b^j$ $s_k<r_1$ $\forall$ $k\in\mathbb{N}$. Then 
$s_k + (r_2-r_1)< r_2$ $\forall k\in\mathbb{N}$. We consider the least $k$ such that $1/10_b^k<(r_2-r_1)$, which exists
since $s_k$ and $r_2-r_1$ are rationals: then $s_k + 1/10_b^k<s_k+(r_2-r_1)<r_2<x$. This yields us 
$$x_k+1< 10_b^k(x-(N+\sum^{k-1}_{j=1} \frac{x_j}{10_b^j}))$$ which contradicts the definition of $x_k$. Therefore, $\nexists$
$y<x$ a lower bound of our set $\{s_k\}$.


\noindent Thus, the decimal expansion is well-defined, and the real number $x$ may be derived as the l.u.b. of $N, x_1,\dots$.
By definition this makes the decimal expansion invertible and a bijection.



\newpage
\subsection*{Problem 19}
\noindent Formulate the definition of the \textbf{greatest lower bound} (g.l.b.) of a set of real numbers.
State a g.l.b. property of $\mathbb{R}$ and show it is equivalent to the l.u.b. property of $\mathbb{R}$.

\medskip
\noindent First, definitions:

\subsection*{Definition} $M\in\mathbb{R}$ is a \textbf{lower bound} for a set $S\subset \mathbb{R}$ if each
$s\in S$ satisfies $s\geq M$. The \textbf{greatest lower bound} is the lower bound $M$ such that $M\geq M'$
for each $M'$ a lower bound of $S$.

\medskip
\noindent The \textbf{Greatest Lower Bound Property} of the complete set $\mathbb{R}$ states: If $S$ is a nonempty 
subset of $\mathbb{R}$ and is bounded above then in $\mathbb{R}$ there exists a greatest lower bound of $S$.

\medskip
\noindent Proof of equivalence to l.u.b. property: Assume the greatest lower bound property. Then for a set $S$ we define
$S'={s : -s\in S}$. $S'$ follows the greatest lower bound property, so $\exists M$ where $s\geq M$ $\forall$ $s\in S'$.
Then we argue that $-M$ is the least upper bound of $S$. For each $s'\in S$ we have $-s'\in S'$: ($s+(-s)=0=(-s)+(-(-(s)))
\implies s= -(-(s))$). Then $-s'\geq M\implies -M\geq s$ by translation. 

\medskip
\noindent For each $M'$ an upper bound of $S$ we have $M'\geq s'$
for each $s'\in S$: by translation this gives us $-M'\leq s$ for $s\in S'$ since $-s\in S$. Then $-M'$ is
a lower bound of $S'$ which gives us $-M'\leq M\implies M'\geq -M$ and $-M$ is the least upper bound of $S$ for each 
bounded $S\subset\mathbb{R}$.

\medskip
\noindent The proof in the other direction (l.u.b property $\implies$ g.l.b. property) is the same with the properties,
but I'm copy-pasting it below anyway out of anxiety :(

\medskip
\noindent Assume the least upper bound property. Then for a set $S$ we define
$S'={s : -s\in S}$. $S'$ follows the least upper bound property, so $\exists M$ where $s\geq M$ $\forall$ $s\in S'$.
Then we argue that $-M$ is the least upper bound of $S$. For each $s'\in S$ we have $-s'\in S'$: ($s+(-s)=0=(-s)+(-(-(s)))
\implies s= -(-(s))$). Then $-s'\leq M\implies -M\leq s$ by translation. 

\medskip
\noindent For each $M'$ a lower bound of $S$ we have $M'\leq s'$
for each $s'\in S$: by translation this gives us $-M'\geq s$ for $s\in S'$ since $-s\in S$. Then $-M'$ is
an upper bound of $S'$ which gives us $-M'\geq M\implies M'\leq -M$ and $-M$ is the greatest lower bound of $S$ for each 
bounded $S\subset\mathbb{R}$.


\newpage
\subsection*{Problem 20}
\noindent Prove that limits are unique, i.e., if $(a_n)$ is a sequence of real numbers that converges to a 
real number $b$ and also converges to a real number $b'$, then $b=b'$.

\medskip
\noindent \textbf{Sol:} Assume $\exists$ $b\neq b'$ where $(a_n)$ converges to both $b$ and $b'$. WLOG assume $b<b'$. Then
$\exists$ $c\in\mathbb{R}$ where $c\in (b, b')$. Take $\epsilon = c-b$. Then by the definition of convergence 
$\exists N\in\mathbb{N}$ where for $n\geq N$, $|a_n-b|<\epsilon$. If $a_n-b$ is positive, $|a_n-b|=a_n-b<c-b\implies a_n<c$. 
If $a_n-b$ is negative, nevertheless we have $a_n-b<0<c-b\implies a_n<c$. From here we take $\epsilon'= b'-c$: $\epsilon'$
is positive. Then $a_n<c\implies b'-a_n>b'-c=\epsilon'$ for all $n\geq N$, and since $b'-a_n$ is positive it cannot satisfy the
definition of convergence to $b'$. Thus, $b=b'$ for any two limits of $(a_n)$.







\newpage
\subsection*{Problem 26}
\noindent Let $b(R)$ and $s(R)$ be the number of integer unit cubes in $\mathbb{R}^m$ that intersect the ball
and sphere of radius $R$, centered at the origin.

\indent (a) Let $m=2$ and calculate the limits $$\lim_{R\rightarrow\infty}\frac{s(R)}{b(R)} \text{ and } \lim_{R\rightarrow\infty}\frac{s(R)^2}{b(R)}.$$ 

\medskip
\noindent \textbf{Sol:} We'll calculate $s(R)$ exactly and $b(R)$ using an upper and lower bound, then use the squeeze theorem.
We know $s(R)=b(R)-c(R)$, so we can define sums for $b(R)/4$ and $c(R)/4$ (quadrants) without evaluating them: $$b(R)/4=\sum_{n=0}^{\lfloor R\rfloor}\Big{\lceil}\sqrt{R^2-n^2}\Big{\rceil}$$
$$c(R)/4=\sum_{n=1}^{\lfloor R\rfloor}\Big{\lceil}\sqrt{R^2-n^2}\Big{\rceil}-1$$
Note that we can use the ceiling function here with half-open boxes only. Then $b(R)-c(R)=\lfloor R\rfloor + \lceil R\rceil$. We define upper and lower bounds for $b(R)$:
$$\pi R^2 < b(R) < \pi (R+\sqrt{2})^2$$ We note that the circle with radius $R+\sqrt{2}$ is an outer bound for $b(R)$ since the greatest distance between points in a unit cube in $\mathbb{R}^2$ is $\sqrt{2}$.

Then we have $$\frac{8R}{\pi(R+\sqrt{2})^2}\leq\frac{4(\lfloor R\rfloor +\lceil R\rceil)}{\pi (R+\sqrt{2})^2} \leq\frac{s(R)}{b(R)}\leq \frac{4(\lfloor R\rfloor +\lceil R\rceil)}{\pi R^2}\leq\frac{8(R+1)}{\pi R^2}$$

$$\frac{8}{\pi(R+\sqrt{2})}-\frac{8\sqrt{2}}{\pi(R+\sqrt{2})^2}\leq \frac{s(R)}{b(R)}\leq \frac{8}{\pi R}+ \frac{8}{\pi R^2}$$
$$\lim_{R\rightarrow\infty}\frac{8}{\pi(R+\sqrt{2})}-\frac{8\sqrt{2}}{\pi(R+\sqrt{2})^2}\leq \lim_{R\rightarrow\infty}\frac{s(R)}{b(R)}\leq \lim_{R\rightarrow\infty}\frac{8}{\pi R}+ \frac{8}{\pi R^2}$$
$$0\leq\lim_{R\rightarrow\infty}\frac{s(R)}{b(R)}\leq 0\implies \lim_{R\rightarrow\infty}\frac{s(R)}{b(R)}=0$$

We can now consider $\frac{s(R)^2}{b(R)}$, noting once again $2R\leq\lfloor R\rfloor+\lceil R\rceil\leq 2(R+1)$:
$$\frac{64R^2}{\pi(R+\sqrt{2})^2}\leq\frac{16(\lfloor R\rfloor +\lceil R\rceil)^2}{\pi (R+\sqrt{2})^2} \leq\frac{s(R)^2}{b(R)}\leq \frac{16(\lfloor R\rfloor +\lceil R\rceil)^2}{\pi R^2}\leq\frac{64(R+1)^2}{\pi R^2}$$

$$\frac{64(R+\sqrt{2})^2}{\pi(R+\sqrt{2})^2}-\frac{64(2\sqrt{2}R+2)}{\pi(R+\sqrt{2})^2}\leq\frac{s(R)^2}{b(R)}\leq\frac{64R^2}{\pi R^2}+\frac{64(2R+1)}{\pi R^2}$$
$$\frac{64}{\pi}\leq\lim_{R\rightarrow\infty}\frac{s(R)^2}{b(R)}\leq\frac{64}{\pi}\implies\lim_{R\rightarrow\infty}\frac{s(R)^2}{b(R)}=\frac{64}{\pi}$$

\bigskip

\indent (b) Take $m\geq 3$. What exponent $k$ makes the limit $$\lim_{R\rightarrow\infty}\frac{s(R)^k}{b(R)}$$ interesting?

\medskip
\noindent \textbf{Sol:} For $k=m/m-1$ the limit is interesting: for any $m\in\mathbb{N}$ we slice the sphere into $\lceil R\rceil$ sections, each of which we can
then take the largest and smallest $m-1$-sphere slices and compare $b(R)$ and $c(R)$. In this case $s(R)$ becomes a degree-$m-1$ function of $\lfloor R\rfloor$ and 
$\lceil R\rceil$. Then $b(R)$ is bounded by $R^m$ and $(R+\sqrt{m})^m$, and so the degrees of the numerator and denominator are the same if $k=\frac{m}{m-1}$, which
yields us nonzero finite bounds for the limit.

\bigskip

\indent (c) Let $c(R)$ be the number of integer unit cubes that are contained in the ball of radius $R$, centered at the origin.
Calculate $$\lim_{R\rightarrow\infty} \frac{c(R)}{b(R)}.$$

\medskip
\noindent \textbf{Sol:} $\lim\limits_{R\rightarrow\infty} \frac{c(R)}{b(R)}=\lim\limits_{R\rightarrow\infty} 1-\frac{s(R)}{b(R)}=1-0=1$. 

\bigskip

\indent (d) Shift the ball to a new, arbitrary center (not on the integer lattice) and re-calculate the limits.

We note that it does not matter in what cube the center is but rather the location of the center within the cube. We can still divide the ball into four
quadrants and exclude the cubes that contain the $y$- or $x$-value of the center of the ball, $(x_0, y_0)$ (WLOG assume $x_0, y_0\in[0,1)^2$). From there we have 
$$b(R)-c(R)=4+2\lceil R+y_0\rceil +2\lceil R-y_0\rceil +2\lfloor R+x_0\rfloor +2\lfloor R-x_0\rfloor$$ which is once again bounded between $8R+4$ and $8R+12$. 
We take the same bounds for $b(R)$ to get $$\frac{8R+4}{\pi(R+\sqrt{2})^2}\leq \frac{s(R)}{b(R)}\leq \frac{8R+12}{\pi R^2}$$ and 
$$\frac{(8R+4)^2}{\pi(R+\sqrt{2})^2}\leq \frac{s(R)}{b(R)}\leq \frac{(8R+12)^2}{\pi R^2}$$ After some calculation we achieve $\lim_{R\rightarrow\infty}\frac{s(R)}{b(R)}=0$ and $\lim_{R\rightarrow\infty}\frac{s(R)^2}{b(R)}=\frac{64}{\pi}$. 

(In general, since the difference between interior and exterior cube numbers remains a function of a rounded $R$-value through the height difference of interior/exterior of adjacent columns,
the ratios and R-degrees remain the same regardless of the center's location.)

\end{document}

