\documentclass{amsart}
\usepackage{../../../lucas}
\usepackage{amsmath, amssymb}
\usepackage{graphicx}

\title{Problem Set 2}
\author{Lucas Chen}
\date{\today}
\begin{document}

\maketitle

\textbf{Problems:} 5, 6, 7, 9, 11, 12, 13, 17, 22, 28 and 101.

\subsection*{Problem 5} For $p, q\in S^1$, the unit circle in the plane, let
$$d_a(p, q) = \min\{|\measuredangle(p) - \measuredangle(q)|, 2\pi - |\measuredangle(p) - \measuredangle(q)|\}$$
where $\measuredangle(z)\in[0,2\pi)$ refers to the angle that z makes with the positive x-axis.
Use your geometric talent to prove that $d_a$ is a metric on $S^1$.

\noindent 

\bigskip

\subsection*{Problem 6} For $p,q \in [0,\pi/2)$ let
$$d_s(p, q) = \sin |p - q|.$$
Use your calculus talent to decide whether $d_s$ is a metric.

\noindent

\bigskip

\subsection*{Problem 7} Prove that every convergent sequence $(p_n)$ in a metric space $M$ is bounded, i.e.,
that for some $r>0$, some $q\in M$, and all $n\in \mathbb{N}$, we have $p_n\in M_rq.$

\noindent

\bigskip

\subsection*{Problem 9} A sequence $(x_n)$ in $\mathbb{R}$ \textbf{increases} if $n < m$ implies $x_n\leq x_m$. 
It \textbf{strictly increases} if $n < m$ implies $x_n < x_m$. It decreases or strictly decreases if $n < m$ 
always implies $x_n\geq x_m$ or always implies $x_n > x_m$. A sequence is monotone if it increases or it decreases. 
Prove that every sequence in $\mathbb{R}$ which is monotone and bounded converges in $\mathbb{R}.$

\noindent

\bigskip

\subsection*{Problem 11} Let $(x_n)$ be a sequence in $\mathbb{R}$.

*(a) Prove that $(x_n)$ has a monotone subsequence.

(b) How can you deduce that every bounded sequence in $\mathbb{R}$ has a convergent subsequence?

(c) Infer that you have a second proof of the Bolzano-Weierstrass Theorem in $\mathbb{R}$.

(d) What about the Heine-Borel Theorem?

\noindent

\bigskip

\subsection*{Problem 12} Let $(p_n)$ be a sequence and $f : \mathbb{N} \rightarrow \mathbb{N}$ 
be a bijection. The sequence $(q_k)_{k\in N}$
with $q_k = p_{f(k)}$ is a rearrangement of $(p_n)$.

(a) Are limits of a sequence unaffected by rearrangement? 

(b) What if $f$ is an injection?

(c) A surjection?

\noindent

\bigskip

\subsection*{Problem 13} Assume that $f : M \rightarrow N$ is a function from one metric space to another 
which satisfies the following condition: If a sequence $(p_n)$ in $M$ converges then the sequence 
$(f(p_n))$ in $N$ converges. Prove that $f$ is continuous. [This result improves Theorem 4.]

\noindent

\bigskip

\subsection*{Problem 17} 17. Which capital letters of the Roman alphabet are homeomorphic? Are any
isometric? Explain.


\bigskip

\subsection*{Problem 22} If every closed and bounded subset of a metric space $M$ is compact, does it 
follow that $M$ is complete?

\noindent

\bigskip

\subsection*{Problem 28} A map $f : M \rightarrow N$ is open if for each open set $U \subset M$, the image set $f(U)$ is
open in $N$.

(a) If $f$ is open, is it continuous?

(b) If $f$ is a homeomorphism, is it open?

(c) If $f$ is an open, continuous bijection, is it a homeomorphism?

(d) If $f : R \rightarrow R$ is a continuous surjection, must it be open?

(e) If $f : R \rightarrow R$ is a continuous, open surjection, must it be a homeomorphism?

(f) What happens in (e) if $R$ is replaced by the unit circle $S^1$?

\noindent

\bigskip

\subsection*{Problem 101} Let $\Sigma$ be the set of all infinite sequences of zeroes and ones. For example, $(100111000011111 . . .) \in\Sigma$. Define the metric
$$d(a,b)=\sum \frac{|a_n-b_n|}{2^n}$$
where $a = (a_n)$ and $b = (b_n)$ are points in $\Sigma$.
(a) Prove that $\Sigma$ is compact.
(b) Prove that $\Sigma$ is homeomorphic to the Cantor set.

\noindent

\bigskip


\end{document}

