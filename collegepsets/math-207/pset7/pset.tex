\documentclass{amsart}
\usepackage{../../lucas}
\usepackage{amsmath, amssymb}
\usepackage{graphicx}

\title{Article Template}
\author{Lucas\ Chen}
\date{\today}
\begin{document}

\maketitle

Problems: 5.21, 5.23, 5.44, 5.51, 5.62

\subsection*{Problem 5.21} For \( U \) as described above, assume that \( f \) is second-differentiable everywhere, and \( (D^2 f)_p = 0 \) for all \( p \). What can you say about \( f \)? Generalize to higher-order differentiability.

\medskip \noindent 

\newpage

\subsection*{Problem 5.23} Assume that \( f : [a, b] \times Y \to \mathbb{R}^m \) is continuous, where \( Y \) is an open subset of \( \mathbb{R}^n \), the partial derivatives \( \partial f_i(x, y) / \partial y_j \) exist, and they are continuous. Let \( D_y f \) be the linear transformation \( \mathbb{R}^n \to \mathbb{R}^m \) represented by the \( m \times n \) matrix of partials.

(a) Show that
    \[
    F(y) = \int_a^b f(x, y) \, dx
    \]
    is of class \( C^1 \) and
    \[
    (DF)_y = \int_a^b (D_y f) \, dx.
    \]
    This generalizes Theorem 14 to higher dimensions.

(b) Generalize (a) to higher-order differentiability.

\medskip \noindent

\newpage

\subsection*{Problem 5.44} Let \( S \subset M \) be given.

(a) Define the characteristic function \( \chi_S : M \to \mathbb{R} \).

(b) If \( M \) is a metric space, show that \( \chi_S(x) \) is discontinuous at \( x \) if and only if \( x \) is a boundary point of \( S \).

\medskip \noindent

\newpage

\subsection*{Problem 5.51} A region \( R \) in the plane is of type 1 if there are smooth functions \( g_1 : [a, b] \to \mathbb{R} \), \( g_2 : [a, b] \to \mathbb{R} \) such that \( g_1(x) \leq g_2(x) \) and
\[
R = \{(x, y) : a \leq x \leq b \text{ and } g_1(x) \leq y \leq g_2(x)\}.
\]
\( R \) is of type 2 if the roles of \( x \) and \( y \) can be reversed, and it is a \textbf{simple region} if it is of both type 1 and type 2.

(a) Give an example of a region that is type 1 but not type 2.

(b) Give an example of a region that is neither type 1 nor type 2.

(c) Is every simple region starlike? Convex?

(d) If a convex region is bounded by a smooth simple closed curve, is it simple?

(e) Give an example of a region that divides into three simple subregions but not into two.

(f) If a region is bounded by a smooth simple closed curve $C$ then it need not divide into a finite number of simple subregions. Find an example.

(g) Infer that the standard proof of Green's Formulas for simple regions (as, for example, in J. Stewart's \textit{Calculus} does not immediately carry over to the general planar region $R$ with smooth boundary; i.e., cutting $R$ into simple regions can fail.

(h) Is there a planar region bounded by a smooth simple closed curve such that for every linear coordinate system (i.e., a new pair of axes), the region does not divine into finitely many simple subregions? In other words, is Stewart's proof of Green's Theorem doomed?

(i) Show that if the curve in (f) is analytic, then no such example exists. [Hint: $C$ is analytic if it is locally the graph of a function defined by convergent power series. A nonconstant analytic function has the property that for each $x$, there is some derivative of $f$ which is nonzero, $f^{(r)}(x)\neq 0$.]

\medskip \noindent

\newpage

\subsection*{Problem 5.62} Does there exist a continuous mapping from the circle to itself that has no fixed-point? What about the 2-torus? The 2-sphere?

\medskip \noindent

\newpage

\end{document}

