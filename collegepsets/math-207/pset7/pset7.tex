\documentclass{amsart}
\usepackage{../../../lucas}
\usepackage{amsmath, amssymb}
\usepackage{graphicx}
\graphicspath{ {./} }

\title{Problem Set 7}
\author{Lucas\ Chen}
\date{\today}
\begin{document}

\maketitle

Problems: 5.21, 5.23, 5.44, 5.51, 5.62

\subsection*{Problem 5.21} For \( U \) as described above, assume that \( f \) is second-differentiable everywhere, and \( (D^2 f)_p = 0 \) for all \( p \). What can you say about \( f \)? Generalize to higher-order differentiability.

\medskip \noindent Assuming the $n^{\text{th}}$ total derivative is $0$ implies that each $n^{\text{th}}$ order
partial derivative is $0$. Thus for component $f_k$ we have $\frac{\partial}{\partial x_{i_1}}\dots\frac{\partial}{\partial x_{i_n}}f_k =0$
for each $i_j\in 1,\dots,m$. Then we have $\frac{\partial}{\partial x_{i_2}}\dots\frac{\partial}{\partial x_{i_n}}f_k= C_{i_2\dots i_n}$ a constant,
since $\frac{\partial}{\partial x_{i_2}}\dots\frac{\partial}{\partial x_{i_n}}f_k$ must be independent of each $x_{i_1}$ (and by 20).

\medskip \noindent From here we will prove by induction that $f$ must be an at most degree $n-1$ polynomial for 
$D^nf=0$. Assume that for $D^{n-1}f=0$ we have $f$ a degree at most $n-2$ polynomial. We know that the order of the $i$s in $C_{i_2\dots i_n}$ 
can be rearranged since rearranging partial variable order results in the same derivative. Take $g=f-\sum C_{?}\frac{x_1^{p_1}\dots x_m^{p_m}}{p_1!\dots p_m!}$,
where the coefficient corresponds to the constant partial ($n-1$)th derivative with respect to each $x_j$ taken $p_j$ times ($\sum p_j = n-1$). (? represents the code corresponding to $p_j$ number of $x_j$s each.)
Then every $(n-1)^{\text{th}}$ partial derivative of $g$ is 0 (since mismatched partial counts lead to the term being differentiated to $0$, and the $(n-1)$th partial of $f$ is $C_?$)
and we know $g$ is a degree at most $n-2$ polynomial. 

\medskip \noindent Since $\sum C_{?}\frac{x_1^{p_1}\dots x_m^{p_m}}{p_1!\dots p_m!}$ is a degree $n-1$ polynomial, we have
that $f= g+\sum C_{?}\frac{x_1^{p_1}\dots x_m^{p_m}}{p_1!\dots p_m!}$ must be a degree $n-1$ polynomial and our inductive hypothesis holds.
Our base case is $n=1$, since we know if $Df=0$ then $f$ is constant (or degree 0). Thus by induction our hypothesis holds for all $n$.

\newpage

\subsection*{Problem 5.23} Assume that \( f : [a, b] \times Y \to \mathbb{R}^m \) is continuous, where \( Y \) is an open subset of \( \mathbb{R}^n \), the partial derivatives \( \partial f_i(x, y) / \partial y_j \) exist, and they are continuous. Let \( D_y f \) be the linear transformation \( \mathbb{R}^n \to \mathbb{R}^m \) represented by the \( m \times n \) matrix of partials.

(a) Show that
    \[
    F(y) = \int_a^b f(x, y) \, dx
    \]
    is of class \( C^1 \) and
    \[
    (DF)_y = \int_a^b (D_y f) \, dx.
    \]
    This generalizes Theorem 14 to higher dimensions.

\medskip \noindent We use the $C^1$ mean value theorem on $f(x, y)$ with respect to $y$. Since $Y$
is open, for some $y_1$ we take the $\epsilon$-neighborhood around $y_1$ and choose $h$ where $|h|<\epsilon$. 
Then the segment between $y_1$ and $y_1+h$ is in $Y$ and we have \[f(x, y_1+h)-f(x, y_1)=\left(\int^1_0(D_yf)_{(x,y_1+th)}\,dt\right)h.\]
Then we take the integral of both sides: \[F(y_1+h)-F(y_1)= \int_a^b\left(\int^1_0(D_yf)_{(x,y_1+th)}\,dt\right)h\,dx\]

Take $D_yF = \int_a^b (D_y f) \, dx$. If \[\lim_{h\to 0} \frac{F(y_1+h)-F(y_1)-(D_yF)_{y_1}h}{|h|} = 0\] then we have verified the derivative.
Since both $F(y_1+h)-F(y_1)$ and $(D_yF)_{y_1}$ are linear transformations we can distribute the scalar divisor
$|h|$ and merge the transformations: \[\lim_{h\to 0}\left(\int_a^b\left(\int^1_0(D_yf)_{(x,y_1+th)}\,dt\right)h\,dx-\int_a^b(D_yf)_{y_1}\,dx\right)\frac{h}{|h|}\]
Then if the transformation is identically $0$ we have proven the derivative. The inner integral is the average derivative over the segment from
$y_1$ to $y_1+h$, which approaches $(D_yf)_{y_1}$ as $h\to 0$, which yields the transformation as the zero transformation. We have 
$D_yF$ continuous since the integral of a continuous function is continuous. 


\bigskip

(b) Generalize (a) to higher-order differentiability.

\medskip \noindent We take an inductive step of the same proof. We must assume that $f$ is $C^n$ and take $D^{n-1}_yf$
as the function whose integral is $D^{n-1}_yF$. Since these derivatives are $(n-1)$-linear transformations we can
fix their inputs and then redo the proof with an additional derivative with respect to $h_n$: the proof then follows the exact
same way replacing $f$ with $D^{n-1}_yf(h_1,\dots h_{n-1})$ since nothing we did required $f$ to map to $R^m$ necessarily (please don't make me write it out please please please :'( )


\newpage

\subsection*{Problem 5.44} Let \( S \subset M \) be given.

(a) Define the characteristic function \( \chi_S : M \to \mathbb{R} \).

\medskip \noindent \[\chi_S(x)=\begin{cases}
    1 & \text{if } x\in S\\
    0 & \text{otherwise.}
\end{cases}
\]

(b) If \( M \) is a metric space, show that \( \chi_S(x) \) is discontinuous at \( x \) if and only if \( x \) is a boundary point of \( S \).

\medskip \noindent Assume first that $x\in S$. If $x$ is a boundary point of $S$, then for each $r>0$ there exists a point $p$
where $p\notin \overline{S}$ but $p\in B_r(x)$. Take $r_n = 1/n$ for $n\in \mathbb{N}$, and the corresponding point
$p_n$. Then $(p_n)$ is a sequence that by definition converges to $x\in S$, but every point of which is mapped
to $0$ by $\chi_S$: thus the sequence $\chi_S(p_n)$ converges to $0$ but $\chi_S(x)=1$ implies discontinuity.

\medskip \noindent Now assume that $x\notin S$. Then if $x$ is a boundary point of $S$ it is in $\overline{S}$, and
is therefore a limit point of a sequence $(x_n)$ in $S$. Once again $\chi_S(x_n)=1$ for all $n$, and
thus the limit of $\chi_S(x_n)$ is $1$: however, $(x_n)$ converges to $x$ and $\chi_S(x)=0$, implying discontinuity.

\medskip \noindent In the other direction, we assume $\chi_S$ is discontinuous at $x$. Then there exists
some sequence $x_n$ that converges to $x$ where $\chi_S(x_n)$ does not converge to $\chi_S(x)$, or where $\chi_S(x_n)$
does not converge. Once again consider $\chi_S(x)=1$: then since $\chi_S(a)=0$ or $\chi_S(a)=1$ for all
$a\in M$, we must have infinitely many $x_k$ where $\chi_S(x_k)=0$, since otherwise we have a maximum $N$ past which
all $x_n$ for $n>N$ must be mapped to $1$. As such, for any $\epsilon>0$ we can pick an $M$ where 
$k>M$ implies $d(x_k, x)<\epsilon$: since $\chi_S(x_k)=0$ we have $x_k\notin S$ for all $k$. Thus $x$ is a boundary point.

\medskip \noindent Now assume $\chi_S(x)=0$. We take the same argument with a convergent sequence $(x_n)$ and infinitely many $k$ for which
$\chi_S(x_k)=1$: then since $x_k\in S$ for all $k$ $(x_k)\to x$ (since it is a subsequence) implies $x$ is a limit point of $S$ and therefore in
the closure of $S$. Thus $x$ is a boundary point.

\newpage

\subsection*{Problem 5.51} A region \( R \) in the plane is of type 1 if there are smooth functions \( g_1 : [a, b] \to \mathbb{R} \), \( g_2 : [a, b] \to \mathbb{R} \) such that \( g_1(x) \leq g_2(x) \) and
\[
R = \{(x, y) : a \leq x \leq b \text{ and } g_1(x) \leq y \leq g_2(x)\}.
\]
\( R \) is of type 2 if the roles of \( x \) and \( y \) can be reversed, and it is a \textbf{simple region} if it is of both type 1 and type 2.

(a) Give an example of a region that is type 1 but not type 2.

\medskip \noindent Take $g_1, g_2: [-2\pi, 2\pi]\to\mathbb{R}$ with $g_1(x)=\sin(x)-2$ and
$g_2(x)=\sin(x)+2$. The points $(\pi/4, 2)$ and $(-3\pi/4, 2)$ are in this set but the horizontal
line between them is not (for instance, $(-\pi/4, 2)$.) Thus the region is not type 2. 

\bigskip

(b) Give an example of a region that is neither type 1 nor type 2.

\medskip \noindent Just take a disconnected set of two points $x_1, y_1$, $x_2, y_2$ with
$x_1\neq x_2$ and $y_1\neq y_2$. Then the set is not type 1 since there is no horizontal interval
of values of $x$ which correspond to points in the set, and the same reason with vertical intervals for type 2.
(Or, take a donut.)

\bigskip


(c) Is every simple region starlike? Convex?

\medskip \noindent Not convex: take the region bounded in $y$ by $y=0$ and $y=x^2$, and in $x$ by $0, 2$. Then the points $(0,0)$ and 
$(2, 4)$ are in the set but the point $(1, 2)$ is not despite being on the segment between them.

\medskip \noindent Not every simple region is starlike :( tried to prove it for twenty minutes

\medskip \noindent Take the simple region bounded by $x^3$ and $x^2$: this is horizontally bounded
by $\sqrt{y}$ and $\sqrt[3]{y}$, each of which is smooth as they are polynomials. The point $(0, 0)$ has
derivative $0$ at both points, so any segment with derivative $>0$ necessarily exceeds $x^3$ initially, and
$y=0$ is less than $x^2$ immediately after $0$. Thus no point is connected by segment to $(0, 0)$ and the
region is not starlike.


\bigskip 

(d) If a convex region is bounded by a smooth simple closed curve, is it simple?

\medskip \noindent Since a smooth simple closed curve is a parametrization $\gamma$ from a compact interval, we can
use the extreme value theorem on the continuous $x$- and $y$-components of the curve to find the 
$t$-values that achieve maximum and minimum of $x$ and $y$. 

\medskip \noindent We show via convexity that the curve has a vertical or horizontal tangent line
on at most two mutually disconnected intervals for each. If there are three vertical tangent lines, then two must
have the same orientation with regard to the parametrization. We assume these two tangent lines
are distinct (because if they are the same with a disconnection between the intervals, then mean value and 
intermediate value theorems imply a third distinct vertical tangent line).

\medskip \noindent Then the line between these points must cross the curve, since both points start in
a vertical direction and begin on opposite sides of the segment. Contradiction. As such there are at most two
vertical and horizontal tangent lines each (due to opposite orientations of the parametrization.) These two must be
the minimum and maximum of $x$ and $y$ respectively. Since the parametrization is smooth, the intermediate value theorem
implies on the intervals between the extrema that the components of the parametrization are monotonic. If we
redo the parametrization with respect to distance (i'm grasping at straws here) then they are strictly monotonic
since the parametrization can't just stop. This implies homeomorphism since the parametrization's components are continuous bijections (by
monotonicity) on compact intervals: since the distance parametrization's components are smooth homeomorphisms and 
strictly monotonic (derivative zero nowhere) by the inverse function theorem they are smooth diffeomorphisms. As a result
the inverse of the $y$-component of the parametrization composed under the $x$-component of the parametrization on the 
interval between $y$-extrema is a smooth function (note that it is not a homeomorphism since $x$ is not monotonic between $y$-extrema!)
and we have the conditions for a simple region satisfied (holy garbage.) 


(Continuous bijection on a compact interval is a homeomorphism: will use later.)

\bigskip




(e) Give an example of a region that divides into three simple subregions but not into two.

\medskip \noindent Take the union of three mutually disconnected simple regions. You are done.

\medskip \noindent If annoying, instead take the union of three rectangles that share only parts of their
boundary. Then this region is connected but any cut of the set into two subregions yields too many corners for
both of the subregions to be simple.

\bigskip

(f) If a region is bounded by a smooth simple closed curve $C$ then it need not divide into a finite number of simple subregions. Find an example.

\medskip \noindent Take any smooth simple closed curve with a smooth transition to the function $f(x)=e^{-\frac{1}{x}}\sin(\frac{1}{x})$. This function
has infinitely many zeros and crosses the $x$-axist across each one, since $e^{-\frac{1}{x}}$ is positive for all $x>0$. Then any attempt
to cut the region into finitely many simple subregions will yield a subregion with more than three zeros, which excludes that subregion from type 2 status.


\bigskip

(g) Infer that the standard proof of Green's Formulas for simple regions (as, for example, in J. Stewart's \textit{Calculus} does not immediately carry over to the general planar region $R$ with smooth boundary; i.e., cutting $R$ into simple regions can fail.

\medskip \noindent $R$ cannot always be cut into simple regions and therefore Stewart's proof of
Green's Theorem does not always have a valid split into useable boundaries and regions. (Essentially,
this follows exactly from (f).)

\bigskip


(h) Is there a planar region bounded by a smooth simple closed curve such that for every linear coordinate system (i.e., a new pair of axes), the region does not divine into finitely many simple subregions? In other words, is Stewart's proof of Green's Theorem doomed?

(i) Show that if the curve in (f) is analytic, then no such example exists. [Hint: $C$ is analytic if it is locally the graph of a function defined by convergent power series. A nonconstant analytic function has the property that for each $x$, there is some derivative of $f$ which is nonzero, $f^{(r)}(x)\neq 0$.]

\medskip \noindent We prove it is impossible for an analytic function to cross a point $a$ infinitely many times on
a compact set. Assume the opposite. Then since the set is compact there is a convergent subsequence whose limit $c$ also
is mapped to $a$ by continuity. By mean value theorem, for each point $x_n$ there is a point between $x_n$ and $c$ whose
derivative is $0$: these points must also converge to $c$ since they are closer than the original sequence, and therefore
since the function is analytic its derivative at $c$ must also be $0$. We continue using the mean value theorem in this way
to conclude that each derivative of the function at $c$ is $0$, which leads to a contradiction since the function is analytic.
Thus the curve cannot cross an $x$ or $y$ value infinitely many times, or on any axis, meaning that the region
is able to be subdivided into finitely many type 1 and type 2 regions, the intersection of which should be finitely many
simple regions. 

\newpage

\subsection*{Problem 5.62} Does there exist a continuous mapping from the circle to itself that has no fixed-point? What about the 2-torus? The 2-sphere?

\medskip \noindent Yes, yes, and yes: for the circle, simply take a rotation of the circle about the origin by some
angle $\neq 2\pi$. For the 2-torus, take the plane parametrization of the torus and translate the plane by some vector
$\neq$ the vector difference between two corners and modulate the outside vectors back into the plane
(this representation is homeomorphic to the torus and thus any fixed points should be preserved.)
For the 2-sphere, map each point to the opposite pole: then no point can be mapped to itself, but
the map is continous (we take $f(x)=-x$.)

\newpage

\end{document}

