\documentclass{amsart}
\usepackage{amssymb}
\usepackage{fancyhdr}
\pagestyle{fancy}

\chead{Lucas Chen}
\cfoot{\thepage}
\rfoot{\today}  


%% your macros -->
\newcommand{\zzz}{\mathbb Z}
\newcommand{\rrr}{\mathbb R}    %% real numbers
\newcommand{\ccc}{\mathbb C}    %% complex numbers
\DeclareMathOperator{\var}{Var}  %% variance

\begin{document}
\Large
\noindent
Assignment {\bf HW 1}   %% <-- Gradescope assignment title here ***

\medskip\noindent
\hrule


\medskip\noindent
Solutions by {\bf Lucas Chen} \qquad   %% <-- your name here ***
  {\tt lucasch(at)uchicago.edu}      %% <-- your uchicago email address here ***

\vspace{0.5cm}

\noindent
This document contains my solutions to
the Gradescope assignment named on the top of this page.
Specifically, my solutions to the following problems
are included:

\vspace{0.4cm}
\begin{itemize}  %% *** update this list
\item 3.35 \quad (page 2)
\item 3.38 (a)(b) \quad (page 3)
\item 3.41 \quad (page 4)
\item 3.44 \quad (page 5)
\item 3.47 (a)(b) \quad (page 6)
\item 3.50 \quad (page 7)
\end{itemize}

\vspace{0.5cm}

\noindent
I did not forget 
\begin{itemize}
  \item to REFRESH my browser for the latest information
   about each problem
  \item \underline{to link problems to pages}.\\
  This page is linked to the problems I did not solve.
\item to update the items marked *** in the template
  (my name, email, the Gradescope title of the assignment,
  the list of problems solved, the \verb>\lhead> statements
  (left page headers: list of (sub)problems solved on each page)
\item to make sure no subproblem solution spills over to the next page
  (except when this is unavoidable, i.e., when the solution to a
  subproblem does not fit on a page)
\item if a problem takes more than one page, I linked
  each of those pages to the problem
\item I took care not to defeat the mechanisms provided by this template.
\end{itemize}
  With each problem, {\bf I stated my sources and collaborations}.\\
  By submitting this solution \emph{I certify} that
  \emph{my statement of sources and collaborations is accurate and complete}.
  I understand that without this certification, my solutions will not be accepted.\\
  In case I am giving a \underline{link} to a source, \emph{I am also sending this link
  to the instructor by email.}
  
\vspace{0.6cm}

\newpage
\Large
\lhead{Problem 3.35}      %% <-- update  ***

\noindent
3.35 Question. \\
Let $F_n$ denote the $n$-th Fibonacci number (see Def. 2.11). Give a very simple proof of the following fact. \\
If the quotients $F_{n+1}/F_n$ converge then their limit is the golden ratio.\\
Do not use the explicit formula for Fibonacci numbers. \\
This is a case when it is easier to compute the limit assuming it exists, than proving the existence of the limit.

\medskip\noindent
\emph{Sources and collaborations.}\\
Worked with Guan Chen on this solution; he suggested solving the quadratic.

\medskip\noindent
\emph{Answer.}\\
We assume there exists a limit. Then for sufficiently high $n$ we have $d(\frac{F_{n+1}}{F_n},\frac{F_{n+2}}{F_{n+1}})<\epsilon$.

\medskip \noindent Take $\frac{F_{n+1}}{F_n}=\frac{F_{n+2}}{F_{n+1}}+a_n$. Then:
\[\frac{F_{n+1}}{F_n}=\frac{F_{n+1}+F_n}{F_{n+1}}+a_n.\]
Take $r_n=\frac{F_{n+1}}{F_n}$. Then we have $r_n = 1+\frac{1}{r_n}+a_n$ with $a_n<\epsilon$ for $n>N_\epsilon$ which yields a quadratic $r_n^2-r_n-1-a_n=0$ which evaluates to \[\frac{1+a_n\pm\sqrt{(1+a_n)^2+4}}{2}\]
and since $\lim_{n\to\infty}a_n=0$, and the ratio must be positive since the Fibonacci sequence is always positive, this approaches $\frac{1+\sqrt{5}}{2}$ and we are done.

\newpage
\lhead{Problem 3.38 (a) (b)}

\noindent
3.38(a) Question.\\
Find two bounded sequences, $(a_n)$ and $(b_n)$, such that $\limsup(a_n+b_n)<\limsup(a_n)+\limsup(b_n)$.

\medskip\noindent
\emph{Sources and collaborations.} 
Worked with Guan Chen.

\medskip\noindent
\emph{Answer.}\\
We define $(a_n)=\frac{(-1)^n}{2}+3/2$ and $(b_n)=\frac{(-1)^{n+1}}{2}+5/2$. Then
$\limsup a_n=2$ and $\limsup b_n = 3$ but $a_n+b_n = 4$ $\forall n$ and the condition is satisfied. \hfill $\Box$

\bigskip\hrule
\vspace{0.5cm}

\noindent
3.38(b) Question.\\
Find two sequences, $(a_n)$ and $(b_n)$, such that $\limsup(a_n+b_n)=-\infty$ while $\limsup a_n=\limsup b_n=\infty$.

\medskip\noindent
\emph{Sources and collaborations.} 
None.

\medskip\noindent
\emph{Answer.} \\
Take the sequences $(a_n) = (1, -2, 2, -4, 3, -6,\dots)$ and \\
$(b_n) = (-2, 1, -4, 2, -6, 3, \dots)$. Then $\sup (a_n)_{n>N} = \infty$ and $\sup (b_n)_{n>N}=\infty$ but $\sup(a_n+b_n)_{n>N}=\lfloor -N/2 \rfloor$. 
\hfill $\Box$

\medskip\noindent

\newpage
\lhead{Problem 3.41}

\medskip\noindent
3.41 Question.\\
Prove $\sqrt{n^2+1}-n \sim 1/(2n)$.

\medskip\noindent
\emph{Sources and collaborations.}
None.

\medskip\noindent
\emph{Answer.} \\
Consider the fraction \[\frac{1/2n}{\sqrt{n^2+1}-n}=\frac{\sqrt{n^2+1}+n}{2n}= \frac{\sqrt{n^2+1}}{2n}+1/2\]
Then \[\lim_{n\to\infty} \frac{\sqrt{n^2+1}}{2n}+1/2 = \sqrt{\lim_{n\to\infty}\frac{n^2+1}{4n^2}}+1/2\]
by positive continuity of $\sqrt{x}$
\[=\sqrt{\lim_{n\to\infty}\frac{n^2}{4n^2}+\lim_{n\to\infty}\frac{1}{4n^2}}+1/2=1/2+1/2=1\]

\newpage
\lhead{Problem 3.44} 

\noindent
3.44 Question.\\
Prove: there exist real numbers $a, b, c$ such that ${2n\choose n} \sim  a \cdot n^b\cdot c^n$. Find $a, b, c$.

\medskip\noindent
\emph{Sources and collaborations.}
Isaac Chang implored me to use my ability to read to see the formula immediately above the problem.

\medskip\noindent
\emph{Answer.} \\
We apply Stirling's Formula. \[{2n\choose n} = \frac{2n!}{n!n!}\] \[n!\sim \left(\frac{n}{e}\right)^n\sqrt{2\pi n} \text{ and } 2n!\sim \left(\frac{2n}{e}\right)^{2n}\sqrt{4\pi n}.\] Then by Exercise 5.3 \[{2n\choose n} \sim \frac{\left(\frac{2n}{e}\right)^{2n}`\sqrt{4\pi n}}{\left(\frac{n}{e}\right)^{2n}2\pi n}\]
\[=\frac{4^{n}}{\sqrt{\pi n}}\]
and we have $a = \frac{1}{\sqrt{\pi}}$, $b= -1/2$, and $c=4$.`

\newpage
\lhead{Problems 3.47 (a)(b)}

\noindent
3.47(a) Question.\\
Assume $a_n, b_n>1$. Consider the following statements:

(A) $a_n \sim b_n$;

(B) $\ln a_n \sim \ln b_n$.

\noindent Prove (A) does not imply (B).

\medskip\noindent
\emph{Sources and collaborations.}
None.

\medskip \noindent 
\emph{Answer.}\\
Take $a_n = \exp(\frac{1}{2^n})$ and $b_n = \exp(\frac{1}{2^{n+1}})$. Then $\frac{a_n}{b_n}=\exp(\frac{1}{2^{n+1}})$ and $\lim_{n\to\infty}\frac{a_n}{b_n} = \exp(0)=1$, but $\frac{\ln a_n}{\ln b_n}=2\neq 1$. Thus (A) cannot imply (B).

\bigskip \hrule
\vspace{0.5cm}

\noindent 
3.47(b) Question.\\
Prove (A) does imply (B) under the stronger assumption that $a_n\geq 1.01$.

\medskip \noindent 
\emph{Sources and collaborations.}
None.

\medskip \noindent 
\emph{Answer.}\\
Assume (A). Then $\lim_{n\to\infty} \frac{a_n}{b_n} = 1$, and since $\ln$ is a continuous
function it preserves limits, yielding \[\lim_{n\to\infty}(\ln a_n - \ln b_n) = 0.\] Since $a_n\geq 1.01$ $\ln(a_n)\geq \ln(1.01)>0$. Then \[\frac{|\ln a_n-\ln b_n|}{\ln a_n}\leq\frac{|\ln a_n-\ln b_n|}{\ln(1.01)}\] and \[\lim_{n\to\infty}\frac{\ln a_n-\ln b_n}{\ln b_n}=\lim_{n\to\infty}\frac{\ln a_n-\ln b_n}{\ln 1.01}= 0\implies \lim_{n\to\infty}\frac{\ln a_n}{\ln b_n} = 1\]

\newpage
\lhead{Problem 3.50}

\noindent
3.50 Question.\\
For the positive integer $m$, let $\nu(m)$ denote the number of distinct prime divisors of $m$. Prove: \[\nu(m)\leq \log_2(m)\]

\medskip\noindent
\emph{Sources and collaborations.}
None.

\medskip \noindent
\emph{Answer.}\\
The number of prime divisors of m can be multiplied to some divisor of $m$ less than $m$, call it $\prod p_j$. Since all primes $\geq 2$, we have $m\geq\prod p_j\geq \prod_{j=1}^{\nu(m)} 2$. Since $\log_2$ is an increasing function, $\log_2 m \geq \nu(m)$ and we are done.


\end{document}
