\documentclass{amsart}
\usepackage{amssymb}
\usepackage{fancyhdr}
\usepackage{algpseudocode}
\pagestyle{fancy}

\chead{Lucas Chen}
\cfoot{\thepage}
\rfoot{\today}  


%% your macros -->
\newcommand{\zzz}{\mathbb Z}
\newcommand{\rrr}{\mathbb R}    %% real numbers
\newcommand{\ccc}{\mathbb C}    %% complex numbers
\DeclareMathOperator{\var}{Var}  %% variance

\begin{document}
\Large
\noindent
Assignment {\bf HW 2}   %% <-- Gradescope assignment title here ***

\medskip\noindent
\hrule


\medskip\noindent
Solutions by {\bf Lucas Chen} \qquad   %% <-- your name here ***
  {\tt lucasch(at)uchicago.edu}      %% <-- your uchicago email address here ***

\vspace{0.5cm}

\noindent
This document contains my solutions to
the Gradescope assignment named on the top of this page.
Specifically, my solutions to the following problems
are included:

\vspace{0.4cm}
\begin{itemize}  %% *** update this list
	\item 5.15 \quad (page 2)
	\item 5.18 \quad (pages 3-4)
	\item 5.22 \quad (page 5)
	\item 5.28 \quad (page 6)
	\item 5.41 \quad (pages 7-8)
	\item 6.15 \quad (pages 9-10)
\end{itemize}

\vspace{0.5cm}

\noindent
I did not forget 
\begin{itemize}
  \item to REFRESH my browser for the latest information
   about each problem
  \item \underline{to link problems to pages}.\\
  This page is linked to the problems I did not solve.
\item to update the items marked *** in the template
  (my name, email, the Gradescope title of the assignment,
  the list of problems solved, the \verb>\lhead> statements
  (left page headers: list of (sub)problems solved on each page)
\item to make sure no subproblem solution spills over to the next page
  (except when this is unavoidable, i.e., when the solution to a
  subproblem does not fit on a page)
\item if a problem takes more than one page, I linked
  each of those pages to the problem
\item I took care not to defeat the mechanisms provided by this template.
\end{itemize}
  With each problem, {\bf I stated my sources and collaborations}.\\
  By submitting this solution \emph{I certify} that
  \emph{my statement of sources and collaborations is accurate and complete}.
  I understand that without this certification, my solutions will not be accepted.\\
  In case I am giving a \underline{link} to a source, \emph{I am also sending this link
  to the instructor by email.}
  
\vspace{0.6cm}

\newpage
\Large
\lhead{Exercise 5.15}      %% <-- update  ***

\noindent
5.15 Question. \\
Let $S(n)>0$ for every $n\in \mathbb{N}$ and assume $S(n)\leq 3S(n/2)+O(n)$ for those $n\geq 2$ that are powers of $2$. From this information we derived that $S(n)=o(n^{\alpha})$ where $\alpha=\log_23\approx 1.58$.

\medskip \noindent Prove that $S(n)=o(n^\alpha)$ does NOT follow from the same information (namely, from the recurrent inequality given and the positivity of $S(n)$.

\medskip\noindent
\emph{Sources and collaborations.}\\


\medskip\noindent
\emph{Answer.}\\
In order to show the desired statement does not follow we may provide a counterexample of some $S(n)$ which satisfies $S(n)\leq 3S(n/2)+O(n)$ but does not satisfy $S(n)=o(n^\alpha)$. The desired counterexample is provided by $S(n)=n^\alpha$ itself: then $\lim_{n\to\infty}\frac{S(n)}{n^\alpha}=\lim_{n\to\infty} 1=1\neq 0$ and $3\left(\frac{n}{2}\right)^\alpha=n^\alpha\geq n^\alpha$, satisfying the condition for $S(n)$ but not for $o(n^\alpha)$. 

\bigskip\hrule

\vspace{0.5cm}

\noindent

\newpage
\lhead{Exercise 5.18}

\noindent
5.18 Question. \\
Let $T(n)$ denote the cost of a Divide-and-Conquer algorithm on inputs of size $n$, so $T(n)>0$ for all $n\in \mathbb{N}$. Assume that the function $T(n)$ is monotone non-decreasing, i.e., $(\forall n\in \mathbb{N})(T(n+1)\geq T(n))$.  Assume further that for all $n\geq 2$ we have $T(n)\leq 7\cdot T(\lceil n/2\rceil)+ O(n^2)$. Prove that $T(n) = O(n^\beta)$ for some constant $\beta$. Find the smallest value of $\beta$ for which this conclusion follows from the assumptions.

\medskip \noindent 
\emph{Sources and collaborations.}\\


\medskip \noindent
\emph{Answer.}\\
We will assume first the stronger condition that $T(n)\leq 7\cdot T(\lceil n/2\rceil)$, find the smallest possible value of $\beta$, and prove it holds for the more general condition. Take $T_1(n) = n^\beta$ for $n$ a power of $2$ (we will fill in the rest of $T_1(n)$ later. We find $\beta$ that satisfies the condition. For some even $n$ we want \[n^\beta\leq 7\left(\frac{n}{2}\right)^\beta\]
and we may set $\beta= \log_2(7)$, yielding $n^\beta\leq n^\beta$. For $n$ not a power of $2$, we set $T_1(n)=T_1(2^m)$ where $2^m$ is the least power of 2 greater than $n$. Then if $2^m<n\leq 2^{m+1}$ we know $2^{m-1}<\frac{\lceil n\rceil}{2}\leq 2^m$ and \[T_1(2^{m+1})=7T_1(2^m)\implies T_1(n)=7T_1(\lceil n/2\rceil)\]

\medskip \noindent We note that since $T_1$ satisfies our stronger condition it satisfies the original condition, and if we take a $\beta<\log_2(7)$, then \[\lim_{m\to\infty} \frac{T_1(2^m)}{2^{m\beta}}= \lim_{m\to\infty}2^{m(\beta-\log_2(7))}\] which diverges.

\medskip \noindent Thus no value of $\beta$ less than $\log_2(7)$ may follow directly from the assumptions (as we cannot rule out $T_1(n)$). 

\medskip \noindent \textbf{Theorem 1}: If $T(n)$ satisfies $T(n)\leq 7\cdot T(\lceil n/2\rceil)+ O(n^2)$ then $T(n)= O(n^\beta)$ where $\beta=\log_2(7)$.


\medskip \noindent \emph{Proof}:
Rewrite the inequality as $T(n)\leq 7\cdot T(\lceil n/2\rceil)+Cn^2$, which we are able to do since $O(n^2)$ is bounded by some positive constant multiple of $n^2$.
We use the method of reverse inequalities. Take $g(n)= An^\beta-Dn^2$ for even $n$ and $g(n)=A(n+1)^\beta-D(n+1)^2$ for odd $n$.  
We aim to find $A$ and $D$ such that \[g(n-1)=g(n)\geq 7\cdot g(n/2)+Cn^2 \geq 7\cdot g(n/2)+C(n-1)^2\]
for even $n$. We solve for 
\[An^\beta-Dn^2\geq 7(A(n/2)^\beta-D(n/2)^2)+Cn^2\]
Then since $n^\beta=7(n/2)^\beta$ this becomes \[\frac{3}{4}Dn^2\geq Cn^2\]
We pick $D=\frac{4}{3}C\geq0$ to satisfy the inequality and $A>D$ so $g(n)>0$ (since $\beta>2$). We take a base case $Kg(1)\geq T(1)$ (and $K>1$). Now assuming $Kg(n)\geq T(n)$ we apply both inductive steps:
\[Kg(2n)\geq 7\cdot Kg(n)+4KCn^2\geq7\cdot T(n)+4Cn^2\geq T(2n)\]
\[Kg(2n-1)\geq 7\cdot Kg(n)+4KCn^2\geq 7\cdot T(n)+C(2n-1)^2\geq T(2n-1)\]
and thus $Kg(n)\geq T(n)$ for all $n\geq 1$. 
We note that \[f(n)=2An^\beta\geq A(n+1)\beta-D(n+1)^2\geq An^\beta-Dn^2\]
meaning $Kf(n)\geq T(n)$ and since $Kf(n), T(n)>0$ this yields $T(n) = O(2An^\beta)=O(n^\beta)$ and we are done.

\newpage
\lhead{Exercise 5.22}

\noindent
5.22 Question. \\
Let $N$ be a $k$-bit positive integer. Prove: $k=\lceil \log_2(N+1)\rceil$. 

\medskip \noindent
\emph{Sources and collaborations.}

\medskip \noindent 
\emph{Answer.}
If $N$ is $k$-bit we must have $2^{k-1}\leq N < 2^k$. Since $\log_2$ is an increasing function this yields:
\[2^{k-1}<N+1\leq 2^k\]
\[k-1 < \log_2(N+1)\leq k\]
and thus $\lceil\log_2(N+1)\rceil = k$ as desired.

\newpage
\lhead{Exercise 5.28}

\noindent
5.28 Question. \\
Find two sequences, $(a_n)$ and $(b_n)$, of positive numbers such that $a_n=\Theta(b_n)$ but the limit $\lim_{n\to\infty}\frac{a_n}{b_n}$ does not exist.

\medskip \noindent
\emph{Sources and collaborations.}

\medskip \noindent 
\emph{Answer.}
We take $a_n = (-1)^n + 2$ and $b_n = 1$. Then $\frac{a_n}{b_n}= a_n$ which diverges as $n$ approaches infinity.

\newpage
\lhead{Exercise 5.41}

\noindent
5.41 Question. \\
(Communication Complexity) Alice has access to an $n$-bit integer $X=\overline{x_0x_1\dots x_{n-1}}$, Bob has access to an $n$-bit integer $Y=\overline{y_0y_1\dots y_{n-1}}$. (Initial zeros are permitted; $x_i, y_i\in \{0,1\}$.) Alice and Bob share a $k$-bit positive integer $q$ such that $X\not\equiv Y$ (mod $q$). The numbers $k$ and $q$ are known to both of them. The task before Alice and Bob is to find $i$ such that $x_i\neq y_i$. Show that they can accomplish this with no more than$\lceil log_2(n)\rceil\cdot (k+1)$ bits of communication. Describe the protocol in \textbf{pseudocode}. 

\medskip \noindent
\emph{Sources and collaborations.}

\medskip \noindent 
\emph{Answer.}

\medskip \noindent Algorithm:

\medskip \noindent 

	\textbf{Binary Search String Comparison}

\medskip INPUT: $X, Y$ $n$-bit binary integers, $q$ an integer. We assume $X\not\equiv Y$ (mod $q$).

\begin{algorithmic}[1]

\Procedure{StringCompare}{$X, Y, q, n$}
\If{$n=1$}
	\Return 0
\EndIf
\State $p:= \lfloor n/2\rfloor$\Comment{Pivot index}
\State $X_2:= X \mod 2^p$\Comment{Second half of X, p bits}
\State $Y_2:= Y \mod 2^p$\Comment{Same for Y}
\If{$X_2\equiv Y_2\mod q$}\Comment{Communication, max $k+1$ bits.} 
	\State\Return StringCompare($\lfloor X/2^p\rfloor, \lfloor Y/2^p\rfloor, q, n-p$)
\Else
`	\State\Return $n-p$ + StringCompare($X_2, Y_2, q, p$)
	\State\Comment{(First index of $X_2$ is $n-p$.)} 
\EndIf
\EndProcedure

\end{algorithmic}


\medskip \noindent 
\emph{Analysis}:
Correctness: procedure ends when $X, Y$ are both one bit long. We verify that StringCompare is never called on $X, Y$ where $X\equiv Y\mod q$: the else statement ensures this trivially. If $X_2$ and $Y_2$ are equivalent then \[X\not\equiv Y\mod q\implies \frac{X-X_2}{2^p}\not\equiv \frac{Y-Y_2}{2^p}\mod q\] and thus the procedure may only end on $1$-bit non-equivalent $X$ and $Y$, when it will recursively return the indices of these bits.

\medskip \noindent Complexity: The only line of communication is the if statement, which requires the mod value of $X_2$ to be communicated (max k bits) and a $1$ or $0$ communicated back for comparison with $Y_2$ (1 bit). This line is run a maximum of $\lceil \log_2(n)\rceil$ times, as described in the binary search handout, as the pivot is calculated the same way — thus the algorithm's communication complexity is $\lceil\log_2(n)\rceil \cdot (k+1)$ bits.
 

\newpage
\lhead{Exercise 6.15}

\noindent
6.15 Question. \\
Given an $n\times n$ array $A$ of zeros and ones, find the maximum size of a contiguous square of all ones. (You do not need to locate such a largest all-ones square, just determine its size.) Solve this problem in \emph{linear time}. "Linear time" means the number of steps must be $O$(size of the input). In the present problem, the size of the input is $O(n^2)$. Manipulating integers between $0$ and $n$ counts as one step; such manipulation includes copying, incrementing, addition ad subtraction, looking up an entry in an $n\times n$ array. Describe your solution in \textbf{pseudocode}.

\medskip \noindent
\emph{Sources and collaborations.}
Initially iterated back over the array $M$ to find the maximum size. Isaac Chang suggested recording this during each step using a variable, which uses almost the same number of integer manipulations (still equivalent to $O(n^2)$) but removes the need to write out two entire nested loops.

\medskip \noindent 
\emph{Answer.}

\medskip \noindent Algorithm:

\medskip \noindent 

	\textbf{Maximum Square Size}

\medskip INPUT: $A$: list[list[int]] of size $n\times n$, $n$ positive integer
\begin{algorithmic}[1]
	\Procedure{MaxSquare}{A, n}
	\State $M :=$ ARRAY[$n$] of ARRAY[$n$] of int\newline\Comment{Entry is largest square in $A$ with bottom left corner at indices.}
	\State $h:=0$\Comment{Variable holding max size}
	\For{$0\leq i<n$}
		\For{$0\leq j<n$}
			\If{$i, j>0$}
			\State $M[i,j]= A[i,j]\cdot 
			\newline(\min(M[i-1,j-1],M[i, j-1],M[i-1,j])$
			\Else
			\State $M[i,j]= A[i,j]$
			\EndIf
			\State $h=\max(h, M[i,j]$
		\EndFor
	\EndFor
	\State \Return $h$
\EndProcedure
\end{algorithmic}

\medskip \noindent \emph{Analysis}

\medskip \noindent Correctness: At each index $[i, j]$ the value of $h$ is the maximum square size in the rectangle with top left at $[0, 0]$ and bottom right at $[i, j]$. The value of $M[i, j]$ is the size of the largest square exactly with its bottom right at $[i, j]$. 
For a square of size $a$ to have a bottom right at $i, j$, there must be squares of at least size $a-1$ with bottom rights in the indices directly above, to the left, and to the upper right diagonal of $[i, j]$, so we check these along with the value of $M[i, j]$. The maximum size square's bottom right corner will inevitably be iterated over so we will inevitably achieve the maximum size in $h$. 

\medskip \noindent Complexity: We count the number of integer manipulations:
\begin{itemize}
	\item Line 3: $1$
	\item Line 4: $n-1$
	\item Line 5, run $n$ times: $n-1$
	\item Line 6, run $n^2$ times: $2$
	\item Lines 7-9, run $n^2$ times: a maximum of $5$
	\item Line 11, run $n^2$ times: $3$
\end{itemize}

\medskip \noindent This yields $11n^2 = O(n^2)$ total integer manipulations and we are done.





\end{document}


