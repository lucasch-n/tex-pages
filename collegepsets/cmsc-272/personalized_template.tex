\documentclass{amsart}
\usepackage{amssymb}
\usepackage{fancyhdr}
\pagestyle{fancy}

\chead{Lucas Chen}
\cfoot{\thepage}
\rfoot{\today}  


%% your macros -->
\newcommand{\zzz}{\mathbb Z}
\newcommand{\rrr}{\mathbb R}    %% real numbers
\newcommand{\ccc}{\mathbb C}    %% complex numbers
\DeclareMathOperator{\var}{Var}  %% variance

\begin{document}
\Large
\noindent
Assignment {\bf HW 1}   %% <-- Gradescope assignment title here ***

\medskip\noindent
\hrule


\medskip\noindent
Solutions by {\bf Lucas Chen} \qquad   %% <-- your name here ***
  {\tt lucasch(at)uchicago.edu}      %% <-- your uchicago email address here ***

\vspace{0.5cm}

\noindent
This document contains my solutions to
the Gradescope assignment named on the top of this page.
Specifically, my solutions to the following problems
are included:

\vspace{0.4cm}
\begin{itemize}  %% *** update this list
\item 4.12 (a)(c)    \quad (page 2)
\item 4.13 (a)(b)(c)  \quad (pages 3--5)
\item 4.17 (a)(b)  \quad (page 6)
\end{itemize}

\vspace{0.5cm}

\noindent
I did not forget 
\begin{itemize}
  \item to REFRESH my browser for the latest information
   about each problem
  \item \underline{to link problems to pages}.\\
  This page is linked to the problems I did not solve.
\item to update the items marked *** in the template
  (my name, email, the Gradescope title of the assignment,
  the list of problems solved, the \verb>\lhead> statements
  (left page headers: list of (sub)problems solved on each page)
\item to make sure no subproblem solution spills over to the next page
  (except when this is unavoidable, i.e., when the solution to a
  subproblem does not fit on a page)
\item if a problem takes more than one page, I linked
  each of those pages to the problem
\item I took care not to defeat the mechanisms provided by this template.
\end{itemize}
  With each problem, {\bf I stated my sources and collaborations}.\\
  By submitting this solution \emph{I certify} that
  \emph{my statement of sources and collaborations is accurate and complete}.
  I understand that without this certification, my solutions will not be accepted.\\
  In case I am giving a \underline{link} to a source, \emph{I am also sending this link
  to the instructor by email.}
  
\vspace{0.6cm}

\newpage
\Large
\lhead{Exercise 4.12 (a) (c)}      %% <-- update  ***

\noindent
4.12(a) Question. \\
Prove the identity
\begin{equation}
  \sum_{k=0}^n \binom{n}{k}^2 = \binom{2n}{n}
\end{equation}

\medskip\noindent
\emph{Sources and collaborations.}\\
Marny Dillon suggested to look up ``Vandermonde's Identity.''\\
in Wikipedia

\medskip\noindent
\emph{Answer.}\\
(your proof here) \hfill $\Box$

\bigskip\hrule

\vspace{0.5cm}

\noindent
4.12 (c)  Question.\\
Prove that in a tree, all longest paths share a vertex.

\medskip\noindent
\emph{Sources and collaborations.} 
I found this at\\
\verb<http://hpca23.cse.wamu.edu/weevil/~jhl/discretebook/chap4.pdf/<\\

\medskip\noindent
\emph{Answer.}\\
(your proof here)  \hfill $\Box$

\newpage
\lhead{Problem 4.13 (a) (b)}

\noindent
4.13 (a) Question.\\
Let $a\in \zzz$.  Let $x=3a-5$ and $y=7a-8$.  Prove:
$\gcd(x,y)$ is either $1$ or $11$.

\medskip\noindent
\emph{Sources and collaborations.} 
Discussed with Marny Dillon.  We figured this out together.

\medskip\noindent
\emph{Answer.}\\
Let $d=\gcd(x,y)$.  Then $d\mid 3y-7x = 11.$  Since 11 is a prime,
its only positive divisors are $1$ and $11$.\hfill $\Box$


\bigskip\hrule
\vspace{0.5cm}

\noindent
4.13(b) Question.\\
Find a value of $a$ such that $\gcd(x,y)=11$.

\medskip\noindent
\emph{Sources and collaborations.} 
Discussed with Marny Dillon.  Marny simplified my more
complicated idea.  Marny suggested the idea of the general
solution; I worked out the details by myself.

\medskip\noindent
\emph{Answer.} \\
Take $a=9$.  Then $x=22$ and $y=55$, so $\gcd(x,y)=11$.
\hfill $\Box$

\medskip\noindent
Comment.  The general solution is
  $$\gcd(x,y)=11 \iff a\equiv -2\pmod{11} .$$

\noindent
\emph{Proof.}  Since $3x-7y=11$, we have that $11\mid x \iff 11\mid y$.
So $\gcd(x,y)=11 \iff 11\mid x \iff 11\mid 3a-5 \iff
3a\equiv 5 \pmod{11}$.  Multiplying both sides by $4$ which
is relatively prime to $11$ we see that
$3a\equiv 5 \pmod{11} \iff 12a\equiv 20 \pmod{11}$.
But $12a\equiv a\pmod{11}$ and $20\equiv -2\pmod{11}$.
\hfill $\Box$

\newpage
\lhead{Problem 4.13 (c)}

\medskip\noindent
4.13(c) Question.\\
For what values $a$ and $b$ is it the case that
$$\gcd(a+b,a-b)=\gcd(a,b)\quad\text{?}$$

\medskip\noindent
\emph{Sources and collaborations.}
None.

\medskip\noindent
\emph{Answer.} \\
For an integer $x$, let $\ell(x)$ denote the largest $k$
such that $2^k \mid x$.  If no largest $k$ exists, we write
$\ell(x)=\infty$.
For instance, $\ell(12)=2$ and $\ell(9)=0$ and $\ell(0)=\infty$.

\medskip\noindent
{\bf Claim.}\quad $\gcd(a+b,a-b)=\gcd(a,b)$ if and only if
either $a=b=0$ or $\ell(a)\neq\ell(b)$.

\medskip\noindent
\emph{Proof.}  If $\gcd(a,b)=0$ then $a=b=0$ and therefore
both sides of the ``if and only if'' statement are true:
$\gcd(a+b,a-b)=\gcd(0,0)=\gcd(a,b)$, and $a=b=0$.

\medskip\noindent
Assume now that $a=0$ and $b\neq 0$.  In this case
again both sides of the ``if and only if'' statement are true:
$\gcd(a,b)=|b|=\gcd(a+b,a-b)$, and $\ell(a)\neq \ell(b)$
because $\ell(a)=\infty$ and $\ell(b)$ is finite.

\medskip\noindent
This also settles the case when $a\neq 0$ and $b=0$ (by switching
the roles of $a$ and $b$).

\medskip\noindent
Henceforth we assume that $a\neq 0$ and $b\neq 0$. \\
In particular, $\gcd(a,b)\neq 0$. 

\medskip\noindent
First we prove the {\bf ``only if'' direction.}
In this part, we have:\\
{\bf Assumption:} \quad $\gcd(a+b,a-b)=\gcd(a,b)$. \\
{\bf Desired conclusion:} $\ell(a)\neq \ell(b)$.\\
\emph{Proof} by contradiction.  Assume for a contradiction that
$\ell(a)=\ell(b)=:k$.  Let $a'=a/2^k$ and $b'=b/2^k$.
Then $\gcd(a,b)=\gcd(2^ka',2^kb')=2^k\gcd(a',b')$
and similarly $\gcd(a+b,a-b)=2^k\gcd(a'+b',a'-b')$.
So our assumption is equivalent to saying that
$\gcd(a'+b',a'-b')=\gcd(a',b')$.

\medskip\noindent
But now both $a'$ and $b'$ are odd, therefore $\gcd(a'b')$
is odd and $\gcd(a'+b',a'-b')$ is even (because both
$a'+b'$ and $a'-b'$ are even), a contradiction
with the assumption that $\gcd(a'+b',a'-b')=\gcd(a',b')$.
This contradiction completes  the proof of the ``only if''
direction.   \hfill $\Box$

\medskip\noindent
Now we prove the {\bf ``if'' direction.}
In this part, we have:\\
{\bf Assumption:} $\ell(a)\neq \ell(b)$.\\
{\bf Desired conclusion:} \quad $\gcd(a+b,a-b)=\gcd(a,b)$. \\

\medskip\noindent
We proceed by first proving a pair of Lemma and a Corollary.

\newpage
\medskip\noindent
{\bf Lemma 1.} For all $a$ and $b$ we have $\gcd(a,b)\mid \gcd(a+b,a-b)$.\\
(Note: ``for all $a$ and $b$'' includes the cases when $a$ or $b$ is zero.)\\
\emph{Proof.}  Let $d\mid a$ and $d\mid b$.  Then (by the additivity of
divisibility) we have $d\mid a+b$ and $d\mid a-b$, and therefore,
$d\mid \gcd(a+b,a-b)$.  \hfill $\Box$

\medskip\noindent
{\bf Lemma 2.} For all $a$ and $b$ we have
 $\gcd(a+b,a-b)\mid 2\cdot\gcd(a,b)$.\\
\emph{Proof.}
Let $D\mid a+b$ and $D\mid a-b$. Then (again by the additivity of
divisibility) we have $D\mid (a+b)+(a-b)=2a$ and $d\mid (a+b)-(a-b)=2b$,
and therefore, $D\mid \gcd(2a,2b)=2\gcd(a,b)$, proving Lemma 2.
\hfill $\Box$

\medskip\noindent
Corollary.  For all $a$ and $b$, the value of $\gcd(a+b,a-b)$ is
either equal to $\gcd(a,b)$ or to $2\cdot\gcd(a,b)$.

\medskip\noindent
First consider the case $\gcd(a,b)=0$.  In this case
$a=b=0$ and therefore $\gcd(a+b,a-b)=\gcd(0,0)=0$.

\medskip\noindent
Assume now that $\gcd(a,b)\neq 0$.
By Lemma 1, there exists an integer $x$ such that 
$\gcd(a+b,a-b)=x\cdot\gcd(a,b)$.  So by Lemma 2,
$x\cdot\gcd(a,b) \mid 2\cdot \gcd(a,b)$.
Since $\gcd(a,b)\neq 0$, we conclude that $x\mid 2$
and therefore $x=\pm1$ or $x=\pm 2$.  Since $x\ge 0$
(because every gcd is by definition $\ge 0$), we conclude
that $x=1$ or $2$, completing the proof of the Corollary.
\hfill $\Box$

\medskip\noindent
Now back to the {\bf proof of the ``if'' direction.}
We continue to assume that $a\neq 0$ and $b\neq 0$.\\
WLOG (without loss of generality)
we may assume that $\ell(a) < \ell(b)$.  Let $k=\ell(a)$ and let
$a'=a/2^k$ and $b'=b/2^k$.  Now $a'$ is odd and $b'$ is even.

\medskip\noindent
As before, we have $\gcd(a,b)=2^k\gcd(a'b')$ and
$\gcd(a+b,a-b)=2^k\gcd(a'+b',a'-b')$.  So to prove our
desired conclusion, it suffices to prove that
$\gcd(a'+b',a'-b')=\gcd(a',b')$.\\
Proof by contradiction.  Assume $\gcd(a'+b',a'-b')\neq\gcd(a',b')$.
Then, by the Corollary, $\gcd(a'+b',a'-b')= 2\cdot\gcd(a',b')$.
This means $\gcd(a'+b',a'-b')$ is even.  But this is impossible
because now both $a'+b'$ and $a'-b'$ are odd.
This contradiction completes the proof of the Claim.
\hfill $\Box$

\newpage
\lhead{Problems 4.17(a)(b)}

\noindent
4.17 (a) Question.\\
Assume $589 \nmid a$.  Does it follow that $a^{588}\equiv 1\pmod{589}$ ?

\medskip\noindent
\emph{Sources and collaborations.}
None.

\medskip\noindent
\emph{Answer.} \\
No.  Counterexample: $a=19$.  Proof by contradiction.  First we
observe that $589 \nmid 19$.  Now
suppose for a contradiction that
$19^{588}\equiv 1 \pmod{589}$.  Then $19^{588}\equiv 1 \pmod{19}$
because $19\mid 589$.  On the other hand, $19^{588}\equiv 0 \pmod{19}$
and therefore $1\equiv 0 \pmod{19}$, a contradiction.
\hfill $\Box$

\bigskip\hrule
\vspace{0.5cm}

\noindent
4.17 (b) Question.\\
Assume $\gcd(a, 589)=1$.  Prove: $a^{90} \equiv 1 \pmod{589}$.

\medskip\noindent
\emph{Sources and collaborations.}
I found a similar problem in Abramov's
``Elementary exercises in number theory,'' Problem 2.17,\\
\verb>http://kvabramov.org/numbook/chap2.pdf>

\medskip\noindent
\emph{Answer.}\\
$589=19\cdot 31$ and both $19$ and $31$ are primes.  In particular,
they are relatively prime; therefore it suffices to prove that
\begin{itemize}
  \item[(i)] $a^{90} \equiv 1 \pmod{19}$ and
  \item[(ii)] $a^{90} \equiv 1 \pmod{31}$.
\end{itemize}
We know that $\gcd(a,19)=1$ and $\gcd(a,31)=1$.  Therefore,
by Fermat's little theorem, we have
\begin{equation}  \label{eq19}
  a^{18}\equiv 1\pmod{19}
\end{equation}
and
\begin{equation}  \label{eq31}
  a^{30}\equiv 1\pmod{31}
\end{equation}

Raising both sides of Eq.~\eqref{eq19} to the fifth power
we get item (i), and similarly, raising both sides of
Eq.~\eqref{eq31} to the third power we obtain item (ii).
\hfill $\Box$

\end{document}
